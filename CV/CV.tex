%!TEX TS-program = xelatex
%!TEX encoding = UTF-8 Unicode
% Awesome CV LaTeX Template for CV/Resume
%
% This template has been downloaded from:
% https://github.com/posquit0/Awesome-CV
%
% Author:
% Claud D. Park <posquit0.bj@gmail.com>
% http://www.posquit0.com
%
%
% Adapted to be an Rmarkdown template by Mitchell O'Hara-Wild
% 23 November 2018
%
% Template license:
% CC BY-SA 4.0 (https://creativecommons.org/licenses/by-sa/4.0/)
%
%-------------------------------------------------------------------------------
% CONFIGURATIONS
%-------------------------------------------------------------------------------
% A4 paper size by default, use 'letterpaper' for US letter
\documentclass[11pt,a4paper,]{awesome-cv}

% Configure page margins with geometry
\usepackage{geometry}
\geometry{left=1.4cm, top=.8cm, right=1.4cm, bottom=1.8cm, footskip=.5cm}


% Specify the location of the included fonts
\fontdir[fonts/]

% Color for highlights
% Awesome Colors: awesome-emerald, awesome-skyblue, awesome-red, awesome-pink, awesome-orange
%                 awesome-nephritis, awesome-concrete, awesome-darknight

\definecolor{awesome}{HTML}{3f51b5}

% Colors for text
% Uncomment if you would like to specify your own color
% \definecolor{darktext}{HTML}{414141}
% \definecolor{text}{HTML}{333333}
% \definecolor{graytext}{HTML}{5D5D5D}
% \definecolor{lighttext}{HTML}{999999}

% Set false if you don't want to highlight section with awesome color
\setbool{acvSectionColorHighlight}{true}

% If you would like to change the social information separator from a pipe (|) to something else
\renewcommand{\acvHeaderSocialSep}{\quad\textbar\quad}

\def\endfirstpage{\newpage}

%-------------------------------------------------------------------------------
%	PERSONAL INFORMATION
%	Comment any of the lines below if they are not required
%-------------------------------------------------------------------------------
% Available options: circle|rectangle,edge/noedge,left/right

\name{João Cascalheira}{}

\position{Researcher}
\address{ICArEHB, Universidade do Algarve, Campus de Gambelas, 8005-139
Faro, Portugal}

\mobile{+351 912 468 699}
\email{\href{mailto:jmcascalheira@ualg.pt}{\nolinkurl{jmcascalheira@ualg.pt}}}
\homepage{www.joaocascalheira.com}
\github{jmcascalheira}
\twitter{joaocascalheira}

% \gitlab{gitlab-id}
% \stackoverflow{SO-id}{SO-name}
% \skype{skype-id}
% \reddit{reddit-id}


\usepackage{booktabs}

\providecommand{\tightlist}{%
	\setlength{\itemsep}{0pt}\setlength{\parskip}{0pt}}

%------------------------------------------------------------------------------



% Pandoc CSL macros
% definitions for citeproc citations
\NewDocumentCommand\citeproctext{}{}
\NewDocumentCommand\citeproc{mm}{%
  \begingroup\def\citeproctext{#2}\cite{#1}\endgroup}
\makeatletter
 % allow citations to break across lines
 \let\@cite@ofmt\@firstofone
 % avoid brackets around text for \cite:
 \def\@biblabel#1{}
 \def\@cite#1#2{{#1\if@tempswa , #2\fi}}
\makeatother
\newlength{\cslhangindent}
\setlength{\cslhangindent}{1.5em}
\newlength{\csllabelwidth}
\setlength{\csllabelwidth}{3em}
\newenvironment{CSLReferences}[2] % #1 hanging-indent, #2 entry-spacing
 {\begin{list}{}{%
  \setlength{\itemindent}{0pt}
  \setlength{\leftmargin}{0pt}
  \setlength{\parsep}{0pt}
  % turn on hanging indent if param 1 is 1
  \ifodd #1
   \setlength{\leftmargin}{\cslhangindent}
   \setlength{\itemindent}{-1\cslhangindent}
  \fi
  % set entry spacing
  \setlength{\itemsep}{#2\baselineskip}}}
 {\end{list}}

\usepackage{calc}
\newcommand{\CSLBlock}[1]{\hfill\break\parbox[t]{\linewidth}{\strut\ignorespaces#1\strut}}
\newcommand{\CSLLeftMargin}[1]{\parbox[t]{\csllabelwidth}{\strut#1\strut}}
\newcommand{\CSLRightInline}[1]{\parbox[t]{\linewidth - \csllabelwidth}{\strut#1\strut}}
\newcommand{\CSLIndent}[1]{\hspace{\cslhangindent}#1}

\begin{document}

% Print the header with above personal informations
% Give optional argument to change alignment(C: center, L: left, R: right)
\makecvheader

% Print the footer with 3 arguments(<left>, <center>, <right>)
% Leave any of these blank if they are not needed
% 2019-02-14 Chris Umphlett - add flexibility to the document name in footer, rather than have it be static Curriculum Vitae
\makecvfooter
  {August 2024}
    {João Cascalheira~~~·~~~Curriculum Vitae}
  {\thepage}


%-------------------------------------------------------------------------------
%	CV/RESUME CONTENT
%	Each section is imported separately, open each file in turn to modify content
%------------------------------------------------------------------------------



\section{Summary}\label{summary}

I am an archaeologist interested in prehistoric hunter-gatherer
adaptations, specifically in how these are related to processes of
climate and environmental change. I focus on the study of lithic
technological organization and its correlation with various aspects of
ecological dynamics and human behavior. I currently direct fieldwork in
several sites within the context of my ERC Consolidator Grant -
\href{www.finisterra.icarehb.com}{FINISTERRA: Population trajectories
and cultural dynamics of the late Neanderthals in Far Western Eurasia}.
I am also a collaborator in different projects on the Middle and Later
Stone Age in Mozambique and Sudan. My publication record counts with 5
books/edited volumes, 45 publications in peer-review journals, and 49
other publications as first author or in collaboration. I also have 188
public presentations in international professional conferences.
Currently, I am the Director of \href{www.icarehb.com}{ICArEHB}, a
research center for archaeology and the evolution of human behaviour at
\href{www.ualg.pt}{Universidade do Algarve}.

\section{\texorpdfstring{\ul{Education}}{Education}}\label{education}

\begin{cventries}
    \cventry{}{Agregação in Archaeology}{Universidade do Algarve, Portugal}{2024}{}\vspace{-4.0mm}
    \cventry{Thesis: The mediterranean influence on the social networks of the Iberian Solutrean}{Ph.D. in Archaeology}{Universidade do Algarve, Portugal}{2013}{}\vspace{-4.0mm}
    \cventry{Final classification: 19 out of 20. Thesis: Solutrean lithic technology from the Vale Boi rockshelter}{MA in Archaeology}{Universidade do Algarve, Portugal}{2009}{}\vspace{-4.0mm}
    \cventry{Final classification: 16 out of 20}{Licenciatura in Cultural Heritage}{Universidade do Algarve, Portugal}{2007}{}\vspace{-4.0mm}
\end{cventries}

\section{\texorpdfstring{\ul{Employment
History}}{Employment History}}\label{employment-history}

\begin{cventries}
    \cventry{Position funded by Fundação para a Ciência e a Tecnologia}{Assistant Researcher}{Universidade do Algarve, Portugal}{2019-Present}{}\vspace{-4.0mm}
\end{cventries}\begin{cventries}
    \cventry{Position funded by Fundação para Ciência e a Tecnologia}{Post-Doctoral Researcher}{Universidade do Algarve, Portugal}{2014-2019}{}\vspace{-4.0mm}
\end{cventries}

\section{\texorpdfstring{\ul{Publications}}{Publications}}\label{publications}

\ul{Citation metrics -- August 2024: \textbf{1388} citations \textbar{}
\href{https://scholar.google.pt/citations?user=bc9Ul_cAAAAJ}{\emph{Google
Scholar}} H-index: \textbf{21} \textbar{}
\href{\%22https://publons.com/researcher/1736137/joao-cascalheira/\%22}{\emph{Web
of Science}} H-index: \textbf{15}}

\begin{center}\includegraphics{CV_files/figure-latex/publicationsgraph-1} \end{center}

\subsection{\texorpdfstring{\ul{Books and Edited
Volumes}}{Books and Edited Volumes}}\label{books-and-edited-volumes}

\phantomsection\label{refs-5cfe0b5dc3734c54a3119901f1778840}
\begin{CSLReferences}{0}{0}
\bibitem[\citeproctext]{ref-cascalheira_2020d}
\CSLLeftMargin{1. }%
\CSLRightInline{Cascalheira, J., \& Picin, A. (Eds.). (2020).
\emph{Short-Term Occupations in Paleolithic Archaeology: Definition and
Interpretation}. Springer International Publishing.
\url{https://doi.org/10.1007/978-3-030-27403-0}}

\bibitem[\citeproctext]{ref-schmidt_2019a}
\CSLLeftMargin{2. }%
\CSLRightInline{Schmidt, I., Cascalheira, J., Bicho, N., \& Weniger,
G.-C. (Eds.). (2019). \emph{Human adaptations to the Last Glacial
Maximum: The Solutrean and its neighbors.} Cambridge Scholars
Publishing.}

\bibitem[\citeproctext]{ref-cascalheira_2018b}
\CSLLeftMargin{3. }%
\CSLRightInline{Cascalheira, J., \& Bicho, N. (2018). \emph{The impact
of drastic environmental changes in prehistoric hunter-gatherer
adaptations} (Vol. 33). Special Issue in Journal of Quaternary Science.}

\bibitem[\citeproctext]{ref-cascalheira_2012e}
\CSLLeftMargin{4. }%
\CSLRightInline{Cascalheira, J., \& Gonçalves, C. (Eds.). (2012).
\emph{Proceedings from Conference of Young Researchers in Archaeology
(JIA 2011)} (Vol. 1--2). Universidade do Algarve.}

\bibitem[\citeproctext]{ref-cascalheira_2010}
\CSLLeftMargin{5. }%
\CSLRightInline{Cascalheira, J. (2010). \emph{Tecnologia lítica do
abrigo Solutrense de Vale Boi}. UNIARQ.}

\end{CSLReferences}

\subsection{\texorpdfstring{\ul{Articles in Peer-Reviewed
Journals}}{Articles in Peer-Reviewed Journals}}\label{articles-in-peer-reviewed-journals}

\phantomsection\label{refs-dc48c4db13c9df2a4d2453784920a95a}
\begin{CSLReferences}{0}{0}
\bibitem[\citeproctext]{ref-barbieri_2023}
\CSLLeftMargin{1. }%
\CSLRightInline{Barbieri, A., Regala, F. T., Cascalheira, J., \& Bicho,
N. (2023). The sediment at the end of the tunnel: Geophysical research
to locate the Pleistocene entrance of Gruta da Companheira (Algarve,
Southern Portugal). \emph{Archaeological Prospection}, \emph{30}(2),
117--134. \url{https://doi.org/10.1002/arp.1881}}

\bibitem[\citeproctext]{ref-godinho_2023}
\CSLLeftMargin{2. }%
\CSLRightInline{Godinho, R. M., Umbelino, C., Valera, A. C., Carvalho,
A. F., Bicho, N., Cascalheira, J., Gonçalves, C., \& Smith, P. (2023).
Mandibular morphology and the Mesolithic--Neolithic transition in
Westernmost Iberia. \emph{Scientific Reports}, \emph{13}(1, 1), 16648.
\url{https://doi.org/10.1038/s41598-023-42846-z}}

\bibitem[\citeproctext]{ref-belmiro_2023c}
\CSLLeftMargin{3. }%
\CSLRightInline{Belmiro, J., Terradas, X., \& Cascalheira, J. (2023).
Creating frames of reference for chert exploitation during the Late
Pleistocene in Southwesternmost Iberia. \emph{PLOS ONE}, \emph{18}(10),
e0293223. \url{https://doi.org/10.1371/journal.pone.0293223}}

\bibitem[\citeproctext]{ref-mylopotamitaki_2023}
\CSLLeftMargin{4. }%
\CSLRightInline{Mylopotamitaki, D., Harking, F. S., Taurozzi, A. J.,
Fagernäs, Z., Godinho, R. M., Smith, G. M., Weiss, M., Schüler, T.,
McPherron, S. P., Meller, H., Cascalheira, J., Bicho, N., Olsen, J. V.,
Hublin, J.-J., \& Welker, F. (2023). Comparing extraction method
efficiency for high-throughput palaeoproteomic bone species
identification. \emph{Scientific Reports}, \emph{13}(1, 1), 18345.
\url{https://doi.org/10.1038/s41598-023-44885-y}}

\bibitem[\citeproctext]{ref-horta_2022}
\CSLLeftMargin{5. }%
\CSLRightInline{Horta, P., Bicho, N., \& Cascalheira, J. (2022). Lithic
bipolar methods as an adaptive strategy through space and time.
\emph{Journal of Archaeological Science: Reports}, \emph{41}, 103263.
\url{https://doi.org/10.1016/j.jasrep.2021.103263}}

\bibitem[\citeproctext]{ref-hallinan_2022b}
\CSLLeftMargin{6. }%
\CSLRightInline{Hallinan, E., Barzilai, O., Bicho, N., Cascalheira, J.,
Demidenko, Y., Goder-Goldberger, M., Hovers, E., Marks, A., Oron, M., \&
Rose, J. (2022). No direct evidence for the presence of Nubian Levallois
technology and its association with Neanderthals at Shukbah Cave.
\emph{Scientific Reports}, \emph{12}(1), 1204.
\url{https://doi.org/10.1038/s41598-022-05072-7}}

\bibitem[\citeproctext]{ref-carvalho_2022a}
\CSLLeftMargin{7. }%
\CSLRightInline{Carvalho, M., Jones, E. L., Ellis, M. G., Cascalheira,
J., Bicho, N., Meiggs, D., Benedetti, M., Friedl, L., \& Haws, J.
(2022). Neanderthal palaeoecology in the late Middle Palaeolithic of
western Iberia: A stable isotope analysis of ungulate teeth from Lapa do
Picareiro (Portugal). \emph{Journal of Quaternary Science},
\emph{37}(2), 300--319. \url{https://doi.org/10.1002/jqs.3363}}

\bibitem[\citeproctext]{ref-ruther_2022}
\CSLLeftMargin{8. }%
\CSLRightInline{Rüther, P. L., Husic, I. M., Bangsgaard, P., Gregersen,
K. M., Pantmann, P., Carvalho, M., Godinho, R. M., Friedl, L.,
Cascalheira, J., Taurozzi, A. J., Jørkov, M. L. S., Benedetti, M. M.,
Haws, J., Bicho, N., Welker, F., Cappellini, E., \& Olsen, J. V. (2022).
SPIN enables high throughput species identification of archaeological
bone by proteomics. \emph{Nature Communications}, \emph{13}(1), 2458.
\url{https://doi.org/10.1038/s41467-022-30097-x}}

\bibitem[\citeproctext]{ref-hallinan_2022a}
\CSLLeftMargin{9. }%
\CSLRightInline{Hallinan, E., Barzilai, O., Beshkani, A., Cascalheira,
J., Demidenko, Y. E., Goder‐Goldberger, M., Hilbert, Y. H., Hovers, E.,
Marks, A. E., \& Nymark, A. (2022). The nature of Nubian: Developing
current global perspectives on Nubian Levallois technology and the
Nubian complex. \emph{Evolutionary Anthropology}, \emph{31}(5),
227--232. \url{https://doi.org/grnbn7}}

\bibitem[\citeproctext]{ref-cascalheira_2022c}
\CSLLeftMargin{10. }%
\CSLRightInline{Cascalheira, J., Gonçalves, C., \& Maio, D. (2022). The
spatial patterning of Middle Palaeolithic human settlement in
westernmost Iberia. \emph{Journal of Quaternary Science}, \emph{37}(2),
291--299. \url{https://doi.org/10.1002/jqs.3286}}

\bibitem[\citeproctext]{ref-haws_2021b}
\CSLLeftMargin{11. }%
\CSLRightInline{Haws, J., Benedetti, M., Carvalho, M., Ellis, G.,
Pereira, T., Cascalheira, J., Bicho, N., \& Friedl, L. (2021). Human
adaptive responses to climate and environmental change during the
Gravettian of Lapa do Picareiro (Portugal). \emph{Quaternary
International}, \emph{587--588}, 4--18.
\url{https://doi.org/10.1016/j.quaint.2020.08.009}}

\bibitem[\citeproctext]{ref-belmiro_2021}
\CSLLeftMargin{12. }%
\CSLRightInline{Belmiro, J., Bicho, N., Haws, J., \& Cascalheira, J.
(2021). The Gravettian-Solutrean transition in westernmost Iberia: New
data from the sites of Vale Boi and Lapa do Picareiro. \emph{Quaternary
International}, \emph{587--588}, 19--40.
\url{https://doi.org/10.1016/j.quaint.2020.08.027}}

\bibitem[\citeproctext]{ref-taylor_2021}
\CSLLeftMargin{13. }%
\CSLRightInline{Taylor, R., García-Rivero, D., Gonçalves, C.,
Cascalheira, J., \& Bicho, N. (2021). The Early Neolithic at the Muge
Shellmiddens (Portugal): Analysis and Review of the Ceramic Evidence
from Cabeço da Amoreira. \emph{European Journal of Archaeology},
\emph{24}(2), 156--179. \url{https://doi.org/10.1017/eaa.2020.40}}

\bibitem[\citeproctext]{ref-cascalheira_2021b}
\CSLLeftMargin{14. }%
\CSLRightInline{Cascalheira, J., Alcaraz-Castaño, M., Alcolea-González,
J., de Andrés-Herrero, M., Arrizabalaga, A., Aura Tortosa, J. E.,
Garcia-Ibaibarriaga, N., \& Iriarte-Chiapusso, M.-J. (2021).
Paleoenvironments and human adaptations during the Last Glacial Maximum
in the Iberian Peninsula: A review. \emph{Quaternary International},
\emph{581--582}, 28--51.
\url{https://doi.org/10.1016/j.quaint.2020.08.005}}

\bibitem[\citeproctext]{ref-cascalheira_2021}
\CSLLeftMargin{15. }%
\CSLRightInline{Cascalheira, J. (2021). Foragers in the middle Limpopo
Valley: Trade, place-making, and social complexity, written by Tim
Forssman. \emph{Journal of African Archaeology}, \emph{19}(2), 248--249.
\url{https://doi.org/10.1163/21915784-20210009}}

\bibitem[\citeproctext]{ref-haws_2021a}
\CSLLeftMargin{16. }%
\CSLRightInline{Haws, J. A., Benedetti, M. M., Bicho, N. F.,
Cascalheira, J., Ellis, M. G., Carvalho, M. M., Friedl, L., Pereira, T.,
\& Talamo, S. (2021). The early Aurignacian at Lapa do Picareiro really
is that old: A comment on {``The late persistence of the Middle
Palaeolithic and Neandertals in Iberia: A review of the evidence for and
against the {`Ebro Frontier'} model.''} \emph{Quaternary Science
Reviews}, \emph{274}, 107261.
\url{https://doi.org/10.1016/j.quascirev.2021.107261}}

\bibitem[\citeproctext]{ref-haws_2020a}
\CSLLeftMargin{17. }%
\CSLRightInline{Haws, J. A., Benedetti, M. M., Talamo, S., Bicho, N.,
Cascalheira, J., Ellis, M. G., Carvalho, M. M., Friedl, L., Pereira, T.,
\& Zinsious, B. K. (2020). The early Aurignacian dispersal of modern
humans into westernmost Eurasia. \emph{Proceedings of the National
Academy of Sciences}, \emph{117}, 25414--25422.
\url{https://doi.org/10.1073/pnas.2016062117}}

\bibitem[\citeproctext]{ref-mclaughlin_2020}
\CSLLeftMargin{18. }%
\CSLRightInline{McLaughlin, T. R., Gómez-Puche, M., Cascalheira, J.,
Bicho, N., \& Fernández-López de Pablo, J. (2020). Late Glacial and
Early Holocene human demographic responses to climatic and environmental
change in Atlantic Iberia. \emph{Philosophical Transactions of the Royal
Society B: Biological Sciences}, \emph{376}(1816), 20190724.
\url{https://doi.org/10.1098/rstb.2019.0724}}

\bibitem[\citeproctext]{ref-paixao_2019}
\CSLLeftMargin{19. }%
\CSLRightInline{Paixão, E., Marreiros, J., Pereira, T., Gibaja, J.,
Cascalheira, J., \& Bicho, N. (2019). Technology, use-wear and raw
material sourcing analysis of a c. 7500 cal BP lithic assemblage from
Cabeço da Amoreira shellmidden (Muge, Portugal). \emph{Archaeological
and Anthropological Sciences}, 1--21.
\url{https://doi.org/10.1007/s12520-018-0621-y}}

\bibitem[\citeproctext]{ref-horta_2019}
\CSLLeftMargin{20. }%
\CSLRightInline{Horta, P., Cascalheira, J., \& Bicho, N. (2019). The
Role of Lithic Bipolar Technology in Western Iberia's Upper Paleolithic:
The Case of Vale Boi (Southern Portugal). \emph{Journal of Paleolithic
Archaeology}, \emph{2}(2), 134--159.
\url{https://doi.org/10.1007/s41982-019-0022-5}}

\bibitem[\citeproctext]{ref-cascalheira_2019a}
\CSLLeftMargin{21. }%
\CSLRightInline{Cascalheira, J. (2019). Territoriality and the
organization of technology during the Last Glacial Maximum in
southwestern Europe. \emph{PLOS ONE}, \emph{14}(12), e0225828.
\url{https://doi.org/10.1371/journal.pone.0225828}}

\bibitem[\citeproctext]{ref-cascalheira_2018d}
\CSLLeftMargin{22. }%
\CSLRightInline{Cascalheira, J., \& Bicho, N. (2018). Testing the impact
of environmental change on hunter-gatherer settlement organization
during the Upper Paleolithic in western Iberia. \emph{Journal of
Quaternary Science}, \emph{33}(3), 323--334.
\url{https://doi.org/10.1002/jqs.3009}}

\bibitem[\citeproctext]{ref-goncalves_2018a}
\CSLLeftMargin{23. }%
\CSLRightInline{Gonçalves, C., Cascalheira, J., Costa, C., Bárbara, S.,
Matias, R., \& Bicho, N. (2018). Detecting single events in large shell
mounds: A GIS approach to Cabeço da Amoreira, Muge, Central Portugal.
\emph{Journal of Archaeological Science: Reports}, \emph{18},
1000--1010. \url{https://doi.org/10.1016/j.jasrep.2017.11.037}}

\bibitem[\citeproctext]{ref-bicho_2018b}
\CSLLeftMargin{24. }%
\CSLRightInline{Bicho, N., Cascalheira, J., Haws, J., \& Gonçalves, C.
(2018). Middle Stone Age Technologies in Mozambique: A Preliminary Study
of the Niassa and Massingir Regions. \emph{Journal of African
Archaeology}, \emph{16}(1), 60--82.
\url{https://doi.org/10.1163/21915784-20180006}}

\bibitem[\citeproctext]{ref-bicho_2018a}
\CSLLeftMargin{25. }%
\CSLRightInline{Bicho, N., Cascalheira, J., André, L., Haws, J., Gomes,
A., Gonçalves, C., Raja, M., \& Benedetti, M. (2018). Portable art and
personal ornaments from Txina-Txina: A new Later Stone Age site in the
Limpopo River Valley, southern Mozambique. \emph{Antiquity},
\emph{92}(363), E2. \url{https://doi.org/10.15184/aqy.2018.95}}

\bibitem[\citeproctext]{ref-bicho_2018}
\CSLLeftMargin{26. }%
\CSLRightInline{Bicho, N., \& Cascalheira, J. (2018). Global
perspectives on the impact of drastic environmental changes in
hunter‐-gatherer technologies. \emph{Journal of Quaternary Science},
\emph{33}(3), 255--260. \url{https://doi.org/10.1002/jqs.3003}}

\bibitem[\citeproctext]{ref-bicho_2017a}
\CSLLeftMargin{27. }%
\CSLRightInline{Bicho, N., Cascalheira, J., \& Gonçalves, C. (2017).
Early Upper Paleolithic colonization across Europe: Time and mode of the
Gravettian diffusion. \emph{PLoS One}, \emph{12}, e0178506.
\url{https://doi.org/10.1371/journal.pone.0178506}}

\bibitem[\citeproctext]{ref-cascalheira_2017c}
\CSLLeftMargin{28. }%
\CSLRightInline{Cascalheira, J., Bicho, N., \& Gonçalves, C. (2017). A
Google-Based Freeware Solution for Archaeological Field Survey and
Onsite Artifact Analysis. \emph{Advances in Archaeological Practice},
\emph{5}, 328--339. \url{https://doi.org/10.1017/aap.2017.21}}

\bibitem[\citeproctext]{ref-cascalheira_2017}
\CSLLeftMargin{29. }%
\CSLRightInline{Cascalheira, J., Bicho, N., Manne, T., \& Horta, P.
(2017). Cross-scale adaptive behaviors during the Upper Paleolithic in
Iberia: The example of Vale Boi (Southwestern Portugal).
\emph{Quaternary International}, \emph{446}, 17--30.
\url{https://doi.org/10.1016/j.quaint.2017.01.002}}

\bibitem[\citeproctext]{ref-bicho_2017e}
\CSLLeftMargin{30. }%
\CSLRightInline{Bicho, N., Cascalheira, J., Marreiros, J., \& Pereira,
T. (2017). Rapid climatic events and long term cultural change: The case
of the Portuguese Upper Paleolithic. \emph{Quaternary International},
\emph{428}, 3--16. \url{https://doi.org/10.1016/j.quaint.2015.05.044}}

\bibitem[\citeproctext]{ref-bicho_2017b}
\CSLLeftMargin{31. }%
\CSLRightInline{Bicho, N., Cascalheira, J., Gonçalves, C., Umbelino, C.,
Rivero, D. G., \& André, L. (2017). Resilience, replacement and
acculturation in the Mesolithic/Neolithic transition: The case of Muge,
central Portugal. \emph{Quaternary International}, \emph{446}, 31--42.
\url{https://doi.org/10.1016/j.quaint.2016.09.049}}

\bibitem[\citeproctext]{ref-bicho_2016a}
\CSLLeftMargin{32. }%
\CSLRightInline{Bicho, N., Haws, J., Raja, M., Madime, O., Gonçalves,
C., Cascalheira, J., Benedetti, M., Pereira, T., \& Aldeias, V. (2016).
Middle and Late Stone Age of the Niassa region, northern Mozambique.
Preliminary results. \emph{Quaternary International}, \emph{404},
87--99. \url{https://doi.org/10.1016/j.quaint.2015.09.059}}

\bibitem[\citeproctext]{ref-pereira_2016a}
\CSLLeftMargin{33. }%
\CSLRightInline{Pereira, T., Bicho, N., Cascalheira, J., Infantini, L.,
Marreiros, J., Paixão, E., \& Terradas, X. (2016). Territory and abiotic
resources between 33 and 15.6 ka at Vale Boi (SW Portugal).
\emph{Quaternary International}, \emph{412, Part A}, 124--134.
https://doi.org/\url{http://dx.doi.org/10.1016/j.quaint.2015.08.071}}

\bibitem[\citeproctext]{ref-goncalves_2016a}
\CSLLeftMargin{34. }%
\CSLRightInline{Gonçalves, C., Raja, M., Madime, O., Cascalheira, J.,
Haws, J., Matos, D., \& Bicho, N. (2016). Mapping the Stone Age of
Mozambique. \emph{African Archaeological Review}, \emph{33}(1), 1--12.
\url{https://doi.org/10.1007/s10437-016-9212-4}}

\bibitem[\citeproctext]{ref-bicho_2015d}
\CSLLeftMargin{35. }%
\CSLRightInline{Bicho, N., Marreiros, J., Cascalheira, J., Pereira, T.,
\& Haws, J. (2015). Bayesian modeling and the chronology of the
Portuguese Gravettian. \emph{Quaternary International}, \emph{359--360},
499--509. \url{https://doi.org/10.1016/j.quaint.2014.04.040}}

\bibitem[\citeproctext]{ref-cascalheira_2015}
\CSLLeftMargin{36. }%
\CSLRightInline{Cascalheira, J., \& Bicho, N. (2015). On the
Chronological Structure of the Solutrean in Southern Iberia. \emph{PLoS
One}, \emph{10}, e0137308.
\url{https://doi.org/10.1371/journal.pone.0137308}}

\bibitem[\citeproctext]{ref-marreiros_2015a}
\CSLLeftMargin{37. }%
\CSLRightInline{Marreiros, J., Bicho, N., Gibaja, J., Pereira, T., \&
Cascalheira, J. (2015). Lithic technology from the Gravettian of Vale
Boi: New insights into Early Upper Paleolithic human behavior in
Southern Iberian Peninsula. \emph{Quaternary International},
\emph{359--360}, 479--498.
https://doi.org/\url{http://dx.doi.org/10.1016/j.quaint.2014.06.074}}

\bibitem[\citeproctext]{ref-tata_2014}
\CSLLeftMargin{38. }%
\CSLRightInline{Tátá, F., Cascalheira, J., Marreiros, J., Pereira, T.,
\& Bicho, N. (2014). Shell bead production in the Upper Paleolithic of
Vale Boi (SW Portugal): An experimental perspective. \emph{Journal of
Archaeological Science}, \emph{42}, 29--41.
\url{https://doi.org/10.1016/j.jas.2013.10.029}}

\bibitem[\citeproctext]{ref-goncalves_2014}
\CSLLeftMargin{39. }%
\CSLRightInline{Gonçalves, C., Cascalheira, J., \& Bicho, N. (2014).
Shellmiddens as landmarks: Visibility studies on the Mesolithic of the
Muge valley (Central Portugal). \emph{Journal of Anthropological
Archaeology}, \emph{36}, 130--139.
\url{https://doi.org/10.1016/j.jaa.2014.09.011}}

\bibitem[\citeproctext]{ref-bicho_2013a}
\CSLLeftMargin{40. }%
\CSLRightInline{Bicho, N., Cascalheira, J., Marreiros, J., Gonçalves,
C., Pereira, T., \& Dias, R. (2013). Chronology of the Mesolithic
occupation of the Muge valley, central Portugal: The case of Cabeço da
Amoreira. \emph{Quaternary International}, \emph{308--309}, 130--139.
\url{https://doi.org/10.1016/j.quaint.2012.10.049}}

\bibitem[\citeproctext]{ref-bicho_2013b}
\CSLLeftMargin{41. }%
\CSLRightInline{Bicho, N., Manne, T., Marreiros, J., Cascalheira, J.,
Pereira, T., Tátá, F., Évora, M., Gonçalves, C., \& Infantini, L.
(2013). The ecodynamics of the first modern humans in Southwestern
Iberia: The case of Vale Boi, Portugal. \emph{Quaternary International},
\emph{318}, 102--116.
\url{https://doi.org/10.1016/j.quaint.2013.06.029}}

\bibitem[\citeproctext]{ref-cascalheira_2013b}
\CSLLeftMargin{42. }%
\CSLRightInline{Cascalheira, J., \& Bicho, N. (2013). Hunter--gatherer
ecodynamics and the impact of the Heinrich event 2 in Central and
Southern Portugal. \emph{Quaternary International}, \emph{318},
117--127. \url{https://doi.org/10.1016/j.quaint.2013.05.039}}

\bibitem[\citeproctext]{ref-pereira_2013b}
\CSLLeftMargin{43. }%
\CSLRightInline{Pereira, T., Cascalheira, J., Marreiros, J., Almeida,
F., \& Bicho, N. (2013). Variation in quartzite exploitation during the
Upper Palaeolithic of Southwest Iberian Peninsula. \emph{Trabajos de
Prehistoria}, \emph{69}, 232--256.
\url{https://doi.org/10.3989/tp.2012.12090}}

\bibitem[\citeproctext]{ref-cascalheira_2012c}
\CSLLeftMargin{44. }%
\CSLRightInline{Cascalheira, J., Bicho, N., Marreiros, J., Pereira, T.,
Évora, M., Cortés Sánchez, M., Gibaja, J. F., Manne, T., Regala, F.,
Gonçalves, C., \& Monteiro, P. (2012). Vale Boi (Algarve, Portugal) and
the Solutrean in Southwestern Iberia. \emph{Espacio, Tiempo y Forma},
\emph{5}, 455--467.
\href{https://doi.org/10.5944/etf\%20i.5.5376}{https://doi.org/10.5944/etf
i.5.5376}}

\bibitem[\citeproctext]{ref-manne_2012}
\CSLLeftMargin{45. }%
\CSLRightInline{Manne, T., Cascalheira, J., Évora, M., Marreiros, J., \&
Bicho, N. (2012). Intensive subsistence practices at Vale Boi, an Upper
Paleolithic site in southwestern Portugal. \emph{Quaternary
International}, \emph{264}, 83--99.
\url{https://doi.org/10.1016/j.quaint.2012.02.026}}

\end{CSLReferences}

\subsection{\texorpdfstring{\ul{Other
Articles}}{Other Articles}}\label{other-articles}

\phantomsection\label{refs-b327b434dae0cffc712ba6b5984a031e}
\begin{CSLReferences}{0}{0}
\bibitem[\citeproctext]{ref-belmiro_2023}
\CSLLeftMargin{1. }%
\CSLRightInline{Belmiro, J., Cascalheira, J., André, L., Matias, R., \&
Gonçalves, C. (2023). Termoclastos no concheiro mesolítico do Cabeço da
Amoreira (Muge, Portugal). \emph{Revista Cultural Do Concelho de
Salvaterra de Magos}.}

\bibitem[\citeproctext]{ref-nogueira_2023}
\CSLLeftMargin{2. }%
\CSLRightInline{Nogueira, D., Godinho, R. M., Gonçalves, C.,
Cascalheira, J., Bicho, N., \& Umbelino, C. (2023). A importância
antropológica de Muge:160 anos de descobertas e desafios. \emph{Revista
Cultural Do Concelho de Salvaterra de Magos}.}

\bibitem[\citeproctext]{ref-bicho_2022a}
\CSLLeftMargin{3. }%
\CSLRightInline{Bicho, N., Gonçalves, C., Cascalheira, J., Umbelino, C.,
Godinho, R. M., \& Costa, C. (2022). O Vale de Muge no contexto do
Mesolítico Atlântico da Península Ibérica. In \emph{Terra e Sal: Das
antigas sociedades camponesas ao fim dos tempos modernos - estudos
oferecidos a Carlos Tavares da Silva} (pp. 59--74). UNIARQ.}

\bibitem[\citeproctext]{ref-cascalheira_2021e}
\CSLLeftMargin{4. }%
\CSLRightInline{Cascalheira, J., Bicho, N., Gonçalves, C.,
Garcia-Rivero, D., \& Horta, P. (2021). Of space and time: The
non-midden components of the Cabeço da Amoreira Mesolithic shellmound
(Muge, Central Portugal). In D. Boric, D. Antonovic, \& B. Mihailović
(Eds.), \emph{Foraging Assemblages} (Serbian Archaeological Society,
Vol. 1, pp. 162--168).}

\bibitem[\citeproctext]{ref-bicho_2020b}
\CSLLeftMargin{5. }%
\CSLRightInline{Bicho, N., \& Cascalheira, J. (2020). The use of lithic
assemblages for the definition of short-term occupations in
hunter-gatherer prehistory. In A. Picin \& J. Cascalheira (Eds.),
\emph{Short-term occupations in Paleolithic Archaeology} (pp. 19--38).
Springer.}

\bibitem[\citeproctext]{ref-picin_2020}
\CSLLeftMargin{6. }%
\CSLRightInline{Picin, A., \& Cascalheira, J. (2020). Introduction to
Short-term Occupations in Paleolithic Archaeology. In A. Picin \& J.
Cascalheira (Eds.), \emph{Short-term occupations in Paleolithic
Archaeology} (pp. 1--18). Springer.}

\bibitem[\citeproctext]{ref-bicho_2020}
\CSLLeftMargin{7. }%
\CSLRightInline{Bicho, N., \& Cascalheira, J. (2020). Dos Neandertais ao
Homo sapiens. In C. Fiolhais, J. Franco, \& J. Paiva (Eds.),
\emph{História Global de Portugal} (pp. 25--31). Temas e Debates.}

\bibitem[\citeproctext]{ref-cardoso_2020}
\CSLLeftMargin{8. }%
\CSLRightInline{Cardoso, J. L., Cascalheira, J., \& Martins, F. (2020).
A estação Solutrense do Olival do Arneiro (Rio Maior). In \emph{Estudos
Arqueológicos de Oeiras} (Câmara Municipal de Oeiras, Vol. 27, pp.
27--98).}

\bibitem[\citeproctext]{ref-bicho_2020a}
\CSLLeftMargin{9. }%
\CSLRightInline{Bicho, N., Cascalheira, J., \& Haws, J. (2020).
Reflexões sobre o estudo do Paleolítico Médio e Superior em Portugal. In
\emph{Estudos Arqueológicos de Oeiras} (Vol. 27, pp. 9--26). Câmara
Municipal de Oeiras.}

\bibitem[\citeproctext]{ref-belmiro_2020c}
\CSLLeftMargin{10. }%
\CSLRightInline{Belmiro, J., Cascalheira, J., \& Gonçalves, C. (2020).
Uma perspectiva diacrónica da ocupação do concheiro do Cabeço da
Amoreira (Muge, Portugal) a partir da tecnologia lítica. In \emph{Actas
do III Congresso da Associação dos Arqueólogos Portugueses}. AAP.}

\bibitem[\citeproctext]{ref-andre_2019}
\CSLLeftMargin{11. }%
\CSLRightInline{André, L., Cascalheira, J., Gonçalves, C., \& Bicho, N.
(2019). Shell beads production during the LGM: The case of Vale Boi
(Southern Portugal). In \emph{Human adaptations to the Last Glacial
Maximum: The Solutrean and its neighbors} (pp. 491--508). Cambridge
Scholars Publishing.}

\bibitem[\citeproctext]{ref-simonvallejo_2019}
\CSLLeftMargin{12. }%
\CSLRightInline{Simón Vallejo, M. D., Bicho, N., Cortés Sánchez, M.,
Parrilla Giráldez, R., \& Cascalheira, J. (2019). New Solutrean portable
art data from the site of Vale Boi (Algarve, Portugal). In \emph{Human
adaptations to the Last Glacial Maximum: The Solutrean and its
neighbors} (pp. 509--517). Cambridge Scholars Publishing.}

\bibitem[\citeproctext]{ref-haws_2019}
\CSLLeftMargin{13. }%
\CSLRightInline{Haws, J., Benedetti, M., Cascalheira, J., Bicho, N.,
Carvalho, M., Zinsious, B., Ellis, G., \& Friedl, L. (2019). Human
occupation during the Late Pleniglacial at Lapa do Picareiro (Portugal).
In \emph{Human adaptations to the Last Glacial Maximum: The Solutrean
and its neighbors} (pp. 188--213). Cambridge Scholars Publishing.}

\bibitem[\citeproctext]{ref-schmidt_2019}
\CSLLeftMargin{14. }%
\CSLRightInline{Schmidt, I., Bicho, N., Cascalheira, J., \& Weniger,
G.-C. (2019). Introduction. In \emph{Human adaptations to the Last
Glacial Maximum: The Solutrean and its neighbors} (pp. x--xiv).
Cambridge Scholars Publishing.}

\bibitem[\citeproctext]{ref-cascalheira_2018h}
\CSLLeftMargin{15. }%
\CSLRightInline{Cascalheira, J., Schmidt, I., \& Bicho, N. (2018). 3.º
Congresso Internacional sobre o Solutrense. \emph{Al-Madan}, \emph{22},
4.}

\bibitem[\citeproctext]{ref-goncalves_2018}
\CSLLeftMargin{16. }%
\CSLRightInline{Gonçalves, C., André, L., Cascalheira, J., \& Bicho, N.
(2018). Der muschelhaufen von Cabeço da Amoreira in Muge, Portugal.
Ergebnisse der archäologischen arbeiten des letzten jahrzehnts.
\emph{Madrider Mitteilumgen}, \emph{59}, 1--21.}

\bibitem[\citeproctext]{ref-tata_2018}
\CSLLeftMargin{17. }%
\CSLRightInline{Tátá, F., Bicho, N., Carvalho, F., Cascalheira, J.,
Gonçalves, C., Luís, R., Luís, V., Mergulho, R., Oliveira, B., Pacheco,
P., Parreira, R., Paulo, L. M., Pinto, M. J., Regala, J., Santos, A., \&
Veiga-Pires, C. (2018). On Charles Bonnet's path. Classic caves of the
Algarve revisited -- Part I. Poço dos Mouros, Igrejinha dos Soidos,
Caverna do Guiné, Abismo and Ibn Ammar caves. \emph{Trogle}, \emph{7},
36--105.}

\bibitem[\citeproctext]{ref-belmiro_2017}
\CSLLeftMargin{18. }%
\CSLRightInline{Belmiro, J., Cascalheira, J., \& Bicho, N. (2017). O
ínicio do Último Máximo Glacial no Sul de Portugal: Novos dados a partir
do sítio arqueológico de Vale Boi. \emph{Arqueologia Em Portugal 2017:
Estado Da Questão}, 375--384.}

\bibitem[\citeproctext]{ref-horta_2017}
\CSLLeftMargin{19. }%
\CSLRightInline{Horta, P., Cascalheira, J., \& Bicho, N. (2017). Sobre a
definição e interpretação das tecnologias líticas bipolares em contextos
pré‑­históricos. \emph{Proceedings of the 2nd Conference of Associação
dos Arqueólogos Portugueses}, 385--392.}

\bibitem[\citeproctext]{ref-varela_2017}
\CSLLeftMargin{20. }%
\CSLRightInline{Varela, J., Bicho, N., Gonçalves, C., \& Cascalheira, J.
(2017). Análise preliminar dos padrões de localização das grutas com
arqueologia do Centro e Sul de Portugal. \emph{Proceedings of the 2nd
Conference of Associação Dos Arqueólogos Portugueses}, 781--794.}

\bibitem[\citeproctext]{ref-abrunhosa_2017a}
\CSLLeftMargin{21. }%
\CSLRightInline{Abrunhosa, A., Cascalheira, J., Pérez-González, A.,
Arsuaga, J. L., \& Baquedano, E. (2017). The use of digital mobile
technologies for geoarchaeological survey: The examples of the Pinilla
del Valle raw materials project. In \emph{Proceedings of the 12th
International Conference of Archaeological Prospection} (pp. 3--4).
Archaeopress.}

\bibitem[\citeproctext]{ref-cascalheira_2017a}
\CSLLeftMargin{22. }%
\CSLRightInline{Cascalheira, J. (2017). Emily Lena Jones . In Search of
the Broad Spectrum Revolution in Paleolithic Southwest Europe (Springers
Briefs in Archaeology. Cham, Heidelberg, New York, Dordrecht \& London:
Springer, 2016, 91pp., 17 figs, 15 tables, pbk, ISBN 978-3-319-22350-6,
eBook, ISBN 978-3-319-22351-3, DOI 10.1007/978-3-319-22351-3).
\emph{European Journal of Archaeology}, \emph{20}(2), 371--375.}

\bibitem[\citeproctext]{ref-marreiros_2016}
\CSLLeftMargin{23. }%
\CSLRightInline{Marreiros, J., Bicho, N., Gibaja, J., Cascalheira, J.,
\& Pereira, T. (2016). Early Gravettian projectile technology in
Southwestern Iberian Peninsula: The double backed and bipointed
bladelets of Vale Boi (Portugal). In \emph{Multidisciplinary approaches
to the study of Stone Age weaponry} (pp. 147--158). Springer.}

\bibitem[\citeproctext]{ref-costa_2016a}
\CSLLeftMargin{24. }%
\CSLRightInline{Costa, C., Gonçalves, C., Cascalheira, J., Marreiros,
J., Pereira, T., Carvalho, S., Valera, A. C., \& Bicho, N. (2016).
ICArEHB -- Centro Interdisciplinar para a Arqueologia e Evolução do
Comportamento Humano. Um novo polo de investigação em Arqueologia.
\emph{Al-Madan}, \emph{20}(2), 98--101.}

\bibitem[\citeproctext]{ref-umbelino_2016}
\CSLLeftMargin{25. }%
\CSLRightInline{Umbelino, C., Gonçalves, C., Figueiredo, O., Pereira,
T., Cascalheira, J., Marreiros, J., \& Bicho, N. (2016). Human burials
in the Mesolithic of Muge and the origins of social differentiation: The
case of Cabeço da Amoreira Portugal. \emph{Proceedings of the
International Conference on Mesolithic Burials--Rites, Symbols and
Social Organisation of Early Postglacial Communities}, \emph{2},
683--692.}

\bibitem[\citeproctext]{ref-cascalheira_2015h}
\CSLLeftMargin{26. }%
\CSLRightInline{Cascalheira, J., Paixão, E., Marreiros, J., Pereira, T.,
\& Bicho, N. (2015). Preliminary techno-typological analysis of the
lithic materials from the Trench area of Cabeço da Amoreira (Muge,
Central Portugal). In \emph{Muge 150th: The 150th Anniversary of the
Discovery of Mesolithic Shellmiddens} (Vol. 1, pp. 119--133). Cambridge
Scholars Publishing.}

\bibitem[\citeproctext]{ref-pereira_2015}
\CSLLeftMargin{27. }%
\CSLRightInline{Pereira, T., Bicho, N., Cascalheira, J., Gonçalves, C.,
Marreiros, J., \& Paixão, E. (2015). Raw material procurement in Cabeço
da Amoreira. In N. Bicho, T. D. Price, \& E. Cunha (Eds.), \emph{Muge
150th: The 150th Anniversary of the Discovery of Mesolithic
Shellmiddens: Volume 1} (Vol. 1, pp. 147--160). Cambridge Scholars
Publishing.}

\bibitem[\citeproctext]{ref-bicho_2015a}
\CSLLeftMargin{28. }%
\CSLRightInline{Bicho, N., Dias, R., Pereira, T., Cascalheira, J.,
Marreiros, J., Pereira, V., \& Gonçalves, C. (2015). O Mesolítico e o
Neolítico antigo: O caso dos concheiros de Muge. \emph{Proceedings of
the 5º Congresso Do Neolítico Peninsular}, 631--638.}

\bibitem[\citeproctext]{ref-cascalheira_2014}
\CSLLeftMargin{29. }%
\CSLRightInline{Cascalheira, J., Gonçalves, C., \& Bicho, N. (2014).
Smartphones and the use of customized Apps in archaeological projects.
\emph{The SAA Archaeological Record}, \emph{14}, 20--25.}

\bibitem[\citeproctext]{ref-figueiredo_2014}
\CSLLeftMargin{30. }%
\CSLRightInline{Figueiredo, O., Cascalheira, J., Marreiros, J., Pereira,
T., Umbelino, C., \& Bicho, N. (2014). Funerary Contexts: The case study
of the Mesolithic shellmiddens of Muge (Portugal). In F. Foulds, H.
Drinkall, A. Perri, \& D. Clinnick (Eds.), \emph{Wild Things: Recent
advances in Palaeolithic and Mesolithic research} (pp. 119--125). Oxbow
Books.}

\bibitem[\citeproctext]{ref-bicho_2014a}
\CSLLeftMargin{31. }%
\CSLRightInline{Bicho, N., Gonçalves, C., \& Cascalheira, J. (2014). Os
concheiros de Muge e a evolução do Mesolítico português. \emph{Revista
Cultural Do Concelho de Salvaterra de Magos}, \emph{1}, 5--18.}

\bibitem[\citeproctext]{ref-monteiro_2014}
\CSLLeftMargin{32. }%
\CSLRightInline{Monteiro, P., Cascalheira, J., Marreiros, J., Pereira,
T., \& Bicho, N. (2014). Fire as a component of Mesolithic Funerary
Rituals: Charcoal analyses from a burial in Cabeço da Amoreira (Muge,
Portugal). In F. Foulds, H. Drinkall, A. Perri, \& D. Clinnick (Eds.),
\emph{Wild Things: Recent advances in Palaeolithic and Mesolithic
research}. Oxbow Books.}

\bibitem[\citeproctext]{ref-cascalheira_2013}
\CSLLeftMargin{33. }%
\CSLRightInline{Cascalheira, J. (2013). O Solutrense em Portugal:
Novidades do século XXI. In J. Arnaud, A. Martins, \& C. Neves (Eds.),
\emph{Arqueologia em Portugal - 150 anos (Atas do I Congresso da
Associação dos Arqueólogos Portugueses)} (pp. 269--276). Associação dos
Arqueólogos Portugueses.}

\bibitem[\citeproctext]{ref-bicho_2013d}
\CSLLeftMargin{34. }%
\CSLRightInline{Bicho, N., Pereira, T., Gonçalves, C., Cascalheira, J.,
Marreiros, J., \& Dias, R. (2013). Os últimos caçadores-recolectores do
vale do Tejo: Novas perspectivas sobre os concheiros de Muge.
\emph{Setúbal Arqueológica (Proceedings of the International Conference
on the Prehistory of Wetlands)}, \emph{14}, 57--68.}

\bibitem[\citeproctext]{ref-dias_2013}
\CSLLeftMargin{35. }%
\CSLRightInline{Dias, R., Cascalheira, J., Gonçalves, C., Detry, C., \&
Bicho, N. (2013). Preliminary analysis of the spatial relationships
between faunal and lithic remains on the Mesolithic shellmidden of
Cabeço da Amoreira (Muge, Portugal). \emph{Proceedings of the
International Conference Environmental Change and Human Interaction in
the Western Atlantic Façade}, 159--164.}

\bibitem[\citeproctext]{ref-bicho_2012a}
\CSLLeftMargin{36. }%
\CSLRightInline{Bicho, N., Cascalheira, J., \& Marreiros, J. (2012). On
the (L)edge: The Case of Vale Boi Rockshelter (Algarve, Southern
Portugal). In K. A. Bergsvik \& R. Skeates (Eds.), \emph{Caves in
Context - The Cultural Significance of Caves and Rockshelters in
Europe}. Oxbow Books.}

\bibitem[\citeproctext]{ref-cascalheira_2012f}
\CSLLeftMargin{37. }%
\CSLRightInline{Cascalheira, J., \& Gonçalves, C. (2012). Spatial
density analysis and site formation processes at the Mesolithic
shellmidden of Cabeço da Amoreira (Muge, Portugal). \emph{Proceedings of
the IV Jornadas de Jovens Em Investigação Arqueológica}, \emph{1},
75--82.}

\bibitem[\citeproctext]{ref-marreiros_2012}
\CSLLeftMargin{38. }%
\CSLRightInline{Marreiros, J., Cascalheira, J., \& Bicho, N. (2012).
Flake technology from the early Gravettian of Vale Boi (Portugal). In
\emph{Flakes not Blades: The Role of Flake Production at the Onset of
the Upper Palaeolithic in Europe. Wissenschaftliche Schriften des
Neanderthal Museums} (Vol. 5, pp. 11--23).}

\bibitem[\citeproctext]{ref-cascalheira_2012}
\CSLLeftMargin{39. }%
\CSLRightInline{Cascalheira, J. (2012). Review of 'Lombera Hermida and
Ramón Fábregas Valcarce. \emph{PaleoAnthropology}, 239--241.}

\bibitem[\citeproctext]{ref-bicho_2011}
\CSLLeftMargin{40. }%
\CSLRightInline{Bicho, N., Cascalheira, J., Marreiros, J., \& Pereira,
T. (2011). The 2008-2010 excavations of Cabeço da Amoreira, Muge,
Portugal. \emph{Mesolithic Miscelanny}, \emph{21}, 3--13.}

\bibitem[\citeproctext]{ref-gibaja_2011a}
\CSLLeftMargin{41. }%
\CSLRightInline{Gibaja, J., Marreiros, J., Cascalheira, J., Palomo, A.,
Carvalho, A. F., \& Rojo, M. (2011). Análisis traceológico del utillaje
lítico documentado en el asentamiento neolítico de Zafrín (Islas
Chafarinas). Configuración de un programa experimental dirigido al
reconocimiento del uso de los perforadores. \emph{Proceedings of the II
Congreso Internacional de Arqueologia Experimental}, 123--129.}

\bibitem[\citeproctext]{ref-bicho_2011b}
\CSLLeftMargin{42. }%
\CSLRightInline{Bicho, N., Cascalheira, J., Marreiros, J., \& Pereira,
T. (2011). The Upper Paleolithic of southern Portugal: 2006-2011. In
\emph{Bilan quinquenal UISPP 2006-2011}.}

\bibitem[\citeproctext]{ref-bicho_2011a}
\CSLLeftMargin{43. }%
\CSLRightInline{Bicho, N., \& Cascalheira, J. (2011). A Arqueologia como
Ciência Tecnológica. \emph{UAlgzine}, \emph{5}, 4--5.}

\bibitem[\citeproctext]{ref-bicho_2010d}
\CSLLeftMargin{44. }%
\CSLRightInline{Bicho, N., Manne, T., Cascalheira, J., Mendonça, C.,
Évora, M., Pereira, T., \& Gibaja, J. (2010). \emph{O Paleolitico
superior do sudoeste da Península Ibérica: O caso do Algarve}.
219--238.}

\bibitem[\citeproctext]{ref-bicho_2010e}
\CSLLeftMargin{45. }%
\CSLRightInline{Bicho, N., Manne, T., Marreiros, J., Cascalheira, J.,
Mendonça, C., Évora, M., Gibaja, J., \& Pereira, T. (2010). O
Paleolítico Superior do Sudoeste da Península Ibérica: O caso do
Algarve. In \emph{El Paleolítico superior peninsular. Novedades del
siglo XXI} (pp. 215--234). Universitat de Barcelona.}

\bibitem[\citeproctext]{ref-bicho_2010f}
\CSLLeftMargin{46. }%
\CSLRightInline{Bicho, N., Pereira, T., Cascalheira, J., Marreiros, J.,
Pereira, V., Jesus, L., \& Gonçalves, C. (2010). Cabeço da Amoreira,
Muge: Resultados dos trabalhos de 2008 e 2009. In \emph{Promontoria
Monográfica 15 (Os últimos caçadores-recolectores e as primeiras
comunidades produtoras do Sul da Península Ibérica e do norte de
Marrocos)} (pp. 11--17). Universidade do Algarve.}

\bibitem[\citeproctext]{ref-marreiros_2009}
\CSLLeftMargin{47. }%
\CSLRightInline{Marreiros, J., Cascalheira, J., Gibaja, J., \& Bicho, N.
(2009). Caracterización de la industria gravetiense y solutrense de Vale
Boi (Algarve, Portugal). \emph{Proceedings of the IV Encontro de
Arqueologia Do Sudoeste Peninsular}.}

\bibitem[\citeproctext]{ref-cascalheira_2009}
\CSLLeftMargin{48. }%
\CSLRightInline{Cascalheira, J. (2009). Tecnologia lítica solutrense do
abrigo de Vale Boi (Algarve, Portugal): Resultados preliminares.
\emph{Proceedings of the I Jornadas de Jovenes En Investigacion:
Dialogando Con La Cultura Material}, \emph{1}.}

\bibitem[\citeproctext]{ref-cascalheira_2008a}
\CSLLeftMargin{49. }%
\CSLRightInline{Cascalheira, J., Marreiros, J., \& Bicho, N. (2008). As
intervenções arqueológicas de 2006 e 2007 no sítio Paleolítico de Vale
Boi. \emph{Xelb. Actas Do 5º Encontro de Arqueologia Do Algarve},
\emph{8}, 26--36.}

\end{CSLReferences}

\section{\texorpdfstring{\ul{Grants, Awards and
Fellowships}}{Grants, Awards and Fellowships}}\label{grants-awards-and-fellowships}

\ul{Total awarded \textasciitilde2000,000€}

\begin{cvhonors}
    \cvhonor{}{FINISTERRA - Population trajectories and cultural dynamics of Late Neanderthals in Far Western Eurasia. Funded by European Research Council Consolidator Grant}{1.899,696€}{2022-}
    \cvhonor{}{Neanderthals and Early Modern Humans in the Escoural Cave (southern Portugal). Funded by Wenner-Gren Foundation}{18,000€}{2020-}
    \cvhonor{}{The origins and evolution of human cognition and the impact of Southwestern European coastal ecology. Funded by Fundação para a Ciência e Tecnologia}{240,000€}{2018-}
    \cvhonor{}{Discovering ancient societies in central Portugal: the Muge shellmiddens. Funded by Earthwatch Institute}{51,378€}{2018-}
    \cvhonor{}{Investigating the earliest Anatomically Modern Human occupation in Western Algarve. Funded by Institute for Field Research}{21,000€}{2018-}
    \cvhonor{}{Ballistic performance of Upper Paleolithic stone-tipped projectiles: an assessment using computational fluid analysis. Funded by Archaeological Institute of America}{4,467€}{2017-}
    \cvhonor{}{Portuguese National Science Foundation Post-Doctoral Fellowship}{107,280€}{2014-19}
    \cvhonor{}{Portuguese National Science Foundation Ph.D. Fellowship}{47,040€}{2010-13}
    \cvhonor{}{COST Action TD0902 – Submerged Prehistoric Archaeology and Landscapes of the Continental Shelf. Short Term Scientific Mission: The Mediterranean influence in the social networks of the Final Solutrean of Iberia: the cave of Parpalló}{1,200€}{2012}
    \cvhonor{}{Paleoanthropology Society – Travel grant to the Paleoanthropology Annual Meeting, May, 2010, St. Louis, US}{400€}{2010}
    \cvhonor{}{Universidade do Algarve – Merit grant}{1,350€}{2009}
    \cvhonor{}{Portuguese National Science Foundation – Investigation grant within the project PTDC/HAH/64185/2006 – UcaRVaT – “The last hunter-gatherers in the Tejo valey – the Muge shellmiddens”}{8,940€}{2008-09}
\end{cvhonors}

\section{\texorpdfstring{\ul{Main Research
Projects}}{Main Research Projects}}\label{main-research-projects}

\subsection{\texorpdfstring{\ul{As Principal Investigator and
Co-Investigator}}{As Principal Investigator and Co-Investigator}}\label{as-principal-investigator-and-co-investigator}

\begin{cvhonors}
    \cvhonor{}{FINISTERRA - Population trajectories and cultural dynamics of Late Neanderthals in Far Western Eurasia. Funded by European Research Council Consolidator Grant}{}{2022-}
    \cvhonor{}{Neanderthals and Early Modern Humans in the Escoural Cave (southern Portugal). Funded by the Wenner-Gren Foundation}{}{2020-}
    \cvhonor{}{The origins and evolution of human cognition and the impact of Southwestern European coastal ecology. Co-PI with Nuno Bicho. Funded by Fundação para a Ciência e Tecnologia - ALG-01-0145-FEDER-27833}{}{2018-}
    \cvhonor{}{Discovering ancient societies in central Portugal: the Muge shellmiddens. Co-PI with Célia Gonçalves, Lino André and Nuno Bicho. Funded by the Earthwatch Institute}{}{2018-}
    \cvhonor{}{Investigating the Earliest anatomically modern human occupation in western Algarve. Co-PI with Nuno Bicho. Funded by the Institute for Field Research}{}{2018-}
    \cvhonor{}{Ballistic performance of Upper Paleolithic stone-tipped projectiles: an assessment using computational fluid analysis. Funded by the Archaeological Institute of America}{}{2017-}
\end{cvhonors}

\subsection{\texorpdfstring{\ul{As
Collaborator}}{As Collaborator}}\label{as-collaborator}

\begin{cvhonors}
    \cvhonor{}{Are Neanderthals adapted to heavy masticatory and paramasticatory function?. PI: Ricardo Godinho. Funded by the Fundação para a Ciência e Tecnologia - 022.07737.PTDC}{}{2023-}
    \cvhonor{}{The Muge Shellmiddens Project: A new portal to the last hunter-gatherers the Tagus valley, Portugal. PI: Célia Gonçalves. Funded by the Fundação para a Ciência e Tecnologia - ALG-01-0145-FEDER-29680}{}{2018-}
    \cvhonor{}{Climate change during the Quaternary in Inhambane, Southwestern Mozambique, and its role in human evolution. PI: Ana Gomes. Funded by the Fundação para a Ciência e Tecnologia - PTDC/HAR-ARQ/28148/2017}{}{2018-}
    \cvhonor{}{Middle and Late Stone Archaeology in Machampane Valley, Limpopo River basin, southern Mozambique. PI: Nuno Bicho. Funded by the National Geographic Society}{}{2017-18}
    \cvhonor{}{Stone Age Vilankulos: Modern Human Origins Research south of the Rio Save, Mozambique. PI: Jonathan Haws. Funded by Fundação para a Ciência e Tecnologia}{}{2016-}
    \cvhonor{}{Middle Stone Age Archaeology survey in the Limpopo and Elephant rivers valleys, southern Mozambique. PI: Nuno Bicho. Funded by the National Geographic Society/Waitt - W-373-15}{}{2015-16}
    \cvhonor{}{Middle Stone Age archaeology and the origins of modern humans in Southern Mozambique. PI: Nuno Bicho. Funded by the Fundação para a Ciência e Tecnologia - PTDC/EPH-ARQ/4998/2012}{}{2013-15}
    \cvhonor{}{GIS predictive modeling for the discovery of new Mesolithic sites in Central Portugal. PI: Célia Gonçalves
Funded by the Calouste Gulbenkian Foundation}{}{2013-2015}
    \cvhonor{}{The last Neanderthals and the emergence of Modern Humans in Southwestern Iberia. PI: Nuno Bicho. Funded by the Fundação para a Ciência e Tecnologia}{}{2012-14}
    \cvhonor{}{On the edge: The first Modern Humans in Southwestern Iberia and the extinction of Neanderthals. PI: Nuno Bicho. Funding: Wenner-Gren Foundation}{}{2011-12}
    \cvhonor{}{The last hunter-gatherers of Muge (Portugal): the origins of social complexity. PI: Nuno Bicho. Funded by the Fundação para a Ciência e Tecnologia - PTDC/HIS‐ARQ/112156/2009}{}{2012-14}
\end{cvhonors}

\section{\texorpdfstring{\ul{Conference
participation}}{Conference participation}}\label{conference-participation}

\subsection{\texorpdfstring{\ul{Oral
communications}}{Oral communications}}\label{oral-communications}

\phantomsection\label{refs-8be73e082360a9f08535bc18ef768fac}
\begin{CSLReferences}{0}{0}
\bibitem[\citeproctext]{ref-bicho_2024}
\CSLLeftMargin{1. }%
\CSLRightInline{Bicho, N., Cascalheira, J., Haws, J., \& Raja, M.
(2024). \emph{Low-Cost Centripetal Technology in the LSA of Southern
Mozambique}. SAA 2024, New Orleans (US).}

\bibitem[\citeproctext]{ref-ferar_2024}
\CSLLeftMargin{2. }%
\CSLRightInline{Ferar, N., Haws, J., \& Cascalheira, J. (2024).
\emph{Raw Material Selection and Technological Expediency in the Iberian
Middle-Upper Paleolithic Transition}. SAA 2024, New Orleans (US).}

\bibitem[\citeproctext]{ref-sanchez-martinez_2024}
\CSLLeftMargin{3. }%
\CSLRightInline{Sanchez-Martinez, J., Ferar, N., Cascalheira, J., \&
Mora, R. (2024). \emph{The Deconstruction of Technical Behavior:
Assessing the Significance of Low-Cost Technologies in the Upper
Paleolithic}. SAA 2024, New Orleans (US).}

\bibitem[\citeproctext]{ref-belmiro_2024}
\CSLLeftMargin{4. }%
\CSLRightInline{Belmiro, J., Galfi, J., Terradas, X., Bicho, N., \&
Cascalheira, J. (2024). \emph{Local or Exogenous? The Different Facets
of Chert during the Gravettian at Vale Boi (Southwestern Portugal)}. SAA
2024, New Orleans (US).}

\bibitem[\citeproctext]{ref-haws_2024}
\CSLLeftMargin{5. }%
\CSLRightInline{Haws, J., Bicho, N., Cascalheira, J., Raja, M., \&
Carvalho, M. (2024). \emph{Archaeological Survey in the Lower Save River
Valley, Southern Mozambique}. SAA 2024, New Orleans (US).}

\bibitem[\citeproctext]{ref-carvalho_2024}
\CSLLeftMargin{6. }%
\CSLRightInline{Carvalho, M., Friedl, L., Benedetti, M., Cascalheira,
J., \& Haws, J. (2024). \emph{Archaeological Survey in the Lower Save
River Valley, Southern Mozambique}. SAA 2024, New Orleans (US).}

\bibitem[\citeproctext]{ref-alzate_2024}
\CSLLeftMargin{7. }%
\CSLRightInline{Alzate, G., Barbieri, A., \& Cascalheira, J. (2024).
\emph{Neanderthal and carnivore interaction at Escoural cave (Southern
Portugal): A micro-geoarchaeological approach}. EAA 2024, Rome (IT).}

\bibitem[\citeproctext]{ref-haws_2024a}
\CSLLeftMargin{8. }%
\CSLRightInline{Haws, J., Benedetti, M., Carvalho, M., Ellis, G.,
Cascalheira, J., Friedl, L., \& Bicho, N. (2024). \emph{The importance
of calcareous landscapes to our understanding of Late Pleistocene
human-environment interactions in Western Iberia}. EAA 2024, Rome (IT).}

\bibitem[\citeproctext]{ref-ferar_2024a}
\CSLLeftMargin{9. }%
\CSLRightInline{Ferar, N., Reeves, J., Moore, M., \& Cascalheira, J.
(2024). \emph{Establishing experimental baselines to interpret low-cost
technological behavior during the Middle to Upper Paleolithic transition
in Southern Iberia}. EAA 2024, Rome (IT).}

\bibitem[\citeproctext]{ref-belmiro_2024a}
\CSLLeftMargin{10. }%
\CSLRightInline{Belmiro, J., Bicho, N., \& Cascalheira, J. (2024).
\emph{LusoLit from the shelf to the web: An online lithotheque from
South Portugal}. EAA 2024, Rome (IT).}

\bibitem[\citeproctext]{ref-bicho_2023}
\CSLLeftMargin{11. }%
\CSLRightInline{Bicho, N., Cascalheira, J., Haws, J., \& Honegger, M.
(2023). \emph{MSA Technology in Kerma, Sudan: The Development of
Fieldwork Methods for Data Acquisition in Basalt Outcrop Settings}. SAA
2023, Portland, USA.}

\bibitem[\citeproctext]{ref-cascalheira_2023a}
\CSLLeftMargin{12. }%
\CSLRightInline{Cascalheira, J., Belmiro, J., André, L., Matias, R., \&
Gonçalves, C. (2023). \emph{Fire-Cracked Rock in the Mesolithic Shell
Midden of Cabeço da Amoreira (Muge, Central Portugal)}. SAA 2023,
Portland, USA.}

\bibitem[\citeproctext]{ref-cascalheira_2023c}
\CSLLeftMargin{13. }%
\CSLRightInline{Cascalheira, J., \& Barbieri, A. (2023). \emph{The
Middle Paleolithic of the Escoural cave (southern Portugal)}.
Neandertales del fin del Mundo. Nuevas perspectivas para Iberia.,
Seville (ES).}

\bibitem[\citeproctext]{ref-haws_2023}
\CSLLeftMargin{14. }%
\CSLRightInline{Haws, J., Carvalho, M., Benedetti, M., \& Cascalheira,
J. (2023). \emph{The Middle-Upper Paleolithic transition in Lapa do
Picareiro (central Portugal)}. Neandertales del fin del Mundo. Nuevas
perspectivas para Iberia., Seville (ES).}

\bibitem[\citeproctext]{ref-cascalheira_2023d}
\CSLLeftMargin{15. }%
\CSLRightInline{Cascalheira, J., Belmiro, J., Abrunhosa, A., \&
Gonçalves, C. (2023). \emph{An Android-based freeware solution for field
survey and onsite artifact analysis}. CAA 2023, Amsterdam (NL).}

\bibitem[\citeproctext]{ref-nogueira_2023a}
\CSLLeftMargin{16. }%
\CSLRightInline{Nogueira, D., Godinho, R. M., Gaspar, R., Bicho, N.,
Cascalheira, J., Gonçalves, C., \& Umbelino, C. (2023). \emph{The
children of the last hunter-gatherers communities, new skeletons from
Cabeço da Amoreira (Portugal)}. EAA 2023, Belfast, NI.}

\bibitem[\citeproctext]{ref-belmiro_2023b}
\CSLLeftMargin{17. }%
\CSLRightInline{Belmiro, J., Galfi, J., Terradas, X., \& Cascalheira, J.
(2023). \emph{Chert provisioning and use during the Upper Paleolithic in
Southwestern Iberia: The case of Vale Boi}. EAA 2023, Belfast, NI.}

\bibitem[\citeproctext]{ref-goncalves_2023}
\CSLLeftMargin{18. }%
\CSLRightInline{Gonçalves, C., Cascalheira, J., Umbelino, C., Godinho,
R. M., \& Nogueira, D. (2023). \emph{Taken Too Soon: The Context of Two
Child Burials at the Mesolithic Shell Midden of Cabeço da Amoreira
(Muge, Portugal)}. SAA 2023, Portland, US.}

\bibitem[\citeproctext]{ref-mylopotamitaki_2023a}
\CSLLeftMargin{19. }%
\CSLRightInline{Mylopotamitaki, D., Harking, F. S., Taurozzi, A. J.,
Fagernäs, Z., Godinho, R., Smith, G. M., Weiss, M., Schuler, T.,
McPherron, S., Meller, H., Cascalheira, J., Bicho, N., Olsen, J. V.,
Hublin, J.-J., \& Welker, F. (2023). \emph{Comparing extraction method
efficiency for high-throughput palaeoproteomic bone species
identification}. ESHE 2023, Aarhus, DK.}

\bibitem[\citeproctext]{ref-cascalheira_2023b}
\CSLLeftMargin{20. }%
\CSLRightInline{Cascalheira, J. (2023). \emph{Data Stewardship:
Competências, boas práticas e desafios}. 10º Forum GDI, Setubal, PT.}

\bibitem[\citeproctext]{ref-nogueira_2022}
\CSLLeftMargin{21. }%
\CSLRightInline{Nogueira, D., Gaspar, R., Gonçalves, C., Cascalheira,
J., Bicho, N., \& Umbelino, C. (2022). \emph{Multiple lésions osseuses
crâniennes sur un individu de Cabeço da Amoreira Mésolithique final,
Portugal)~: Apport de la paléoradiologie au diagnostic différentiel}.
Colloque du Groupe des Paléopathologistes de Langue Française, Paris,
FR.}

\bibitem[\citeproctext]{ref-bicho_2022}
\CSLLeftMargin{22. }%
\CSLRightInline{Bicho, N., \& Cascalheira, J. (2022). \emph{The Upper
Paleolithic site of Vale Boi, Southern Portugal}. SAA 2022, Chicago,
USA.}

\bibitem[\citeproctext]{ref-barbieri_2022}
\CSLLeftMargin{23. }%
\CSLRightInline{Barbieri, A., Cascalheira, J., Aldeias, V., \& Bicho, N.
(2022). \emph{Upper Paleolithic foragers on a slope: Geoarchaeological
data from Vale Boi}. SAA 2022, Chicago, USA.}

\bibitem[\citeproctext]{ref-cascalheira_2022}
\CSLLeftMargin{24. }%
\CSLRightInline{Cascalheira, J., Belmiro, J., \& Bicho, N. (2022).
\emph{The Last Glacial Maximum at Vale Boi}. SAA 2022, Chicago, USA.}

\bibitem[\citeproctext]{ref-horta_2022a}
\CSLLeftMargin{25. }%
\CSLRightInline{Horta, P., Cascalheira, J., \& Bicho, N. (2022).
\emph{The Recurrent Use of Lithic Bipolar Technology as a Resource
Extraction Strategy during Upper Paleolithic of Vale Boi}. SAA 2022,
Chicago, USA.}

\bibitem[\citeproctext]{ref-carvalho_2022b}
\CSLLeftMargin{26. }%
\CSLRightInline{Carvalho, M., Cascalheira, J., \& Bicho, N. (2022).
\emph{Human Paleoecology during the Gravettian and Proto-Solutrean in
Southwestern Iberia: A Stable Isotope Analysis of Herbivore Teeth from
Vale Boi (Portugal)}. SAA 2022, Chicago, USA.}

\bibitem[\citeproctext]{ref-goncalves_2022}
\CSLLeftMargin{27. }%
\CSLRightInline{Gonçalves, C., Maio, D., Cascalheira, J., \& Bicho, N.
(2022). \emph{Intrasite Spatial Analysis at the Upper Paleolithic Site
of Vale Boi, Southern Portugal}. SAA 2022, Chicago, USA.}

\bibitem[\citeproctext]{ref-belmiro_2022a}
\CSLLeftMargin{28. }%
\CSLRightInline{Belmiro, J., Cascalheira, J., Terradas, X., \& Bicho, N.
(2022). \emph{Raw Material Provisioning and Use throughout the Upper
Paleolithic at Vale Boi (Southwestern Iberia)}. SAA 2022, Chicago, USA.}

\bibitem[\citeproctext]{ref-simonvallejo_2022}
\CSLLeftMargin{29. }%
\CSLRightInline{Simón Vallejo, M. D., Cortés Sánchez, M.,
Parrila-Giraldez, R., Cascalheira, J., \& Bicho, N. (2022).
\emph{Portable Art of the Vale Boi Site (Algarve, Portugal), in the
Context of the Paleolithic Art of the South of the Iberian Peninsula}.
SAA 2022, Chicago, USA.}

\bibitem[\citeproctext]{ref-haws_2022}
\CSLLeftMargin{30. }%
\CSLRightInline{Haws, J., Benedetti, M., Bicho, N., Carvalho, M.,
Friedl, L., Ellis, G., Pereira, T., \& Cascalheira, J. (2022). \emph{The
Magdalenian settlement of Portuguese Estremadura: New data from Lapa do
Picareiro}. EAA 2022, Budapest.}

\bibitem[\citeproctext]{ref-carvalho_2022}
\CSLLeftMargin{31. }%
\CSLRightInline{Carvalho, M., Jones, E. L., Ellis, G., Cascalheira, J.,
Bicho, N., Benedetti, M., Friedl, L., \& Haws, J. (2022).
\emph{Neanderthal ecology at Lapa do Picareiro: A stable isotope study
of ungulate tooth enamel}. EAA 2022, Budapest.}

\bibitem[\citeproctext]{ref-cascalheira_2022b}
\CSLLeftMargin{32. }%
\CSLRightInline{Cascalheira, J., Barbieri, A., Belmiro, J., Carvalho,
M., Galfi, J., Gonçalves, C., \& Bicho, N. (2022). \emph{The
Gravettian-Solutrean transition at Vale Boi (southern Portugal): New
stratigraphic, chronological, and paleoenvironmental data}. Facing the
Last Glacial Maximum: Fresh insights into the Gravettian-Solutrean
transition in Southwestern Europe, Lisbon, PT.}

\bibitem[\citeproctext]{ref-belmiro_2022}
\CSLLeftMargin{33. }%
\CSLRightInline{Belmiro, J., Bicho, N., \& Cascalheira, J. (2022).
\emph{Lithic technology and raw material use during the
Gravettian-Solutrean transition in Southern Portugal}. Facing the Last
Glacial Maximum: Fresh insights into the Gravettian-Solutrean transition
in Southwestern Europe, Lisbon, PT.}

\bibitem[\citeproctext]{ref-cascalheira_2022a}
\CSLLeftMargin{34. }%
\CSLRightInline{Cascalheira, J., André, L., Tátá, F., \& Bicho, N.
(2022). \emph{Vale Boi. Los recursos marinos y de adornos de las
ocupaciones paleolíticas del Algarve (Portugal)}. INNOVAZUL 2022,
Cádis.}

\bibitem[\citeproctext]{ref-maio_2022}
\CSLLeftMargin{35. }%
\CSLRightInline{Maio, D., Gonçalves, C., \& Cascalheira, J. (2022).
\emph{Moving Around: Spatial Patterning of Middle-Upper Paleolithic
Occupation in Portuguese Extremadura}. 7th Landscape Archaeology
Conference, Iasi, RO.}

\bibitem[\citeproctext]{ref-bicho_2021a}
\CSLLeftMargin{36. }%
\CSLRightInline{Bicho, N., Haws, J., Cascalheira, J., Gonçalves, C., \&
Raja, M. (2021). \emph{Stone Age Archaeology in the Elephant River
Valley, Southwestern Mozambique}. SAA 2021, Online.}

\bibitem[\citeproctext]{ref-haws_2021}
\CSLLeftMargin{37. }%
\CSLRightInline{Haws, J., Bicho, N., Cascalheira, J., Raja, M., \&
Carvalho, M. (2021). \emph{Stone Age Archaeology in the Lower Save River
Valley, Southern Mozambique}. SAA 2021, Online.}

\bibitem[\citeproctext]{ref-fonseca_2021}
\CSLLeftMargin{38. }%
\CSLRightInline{Fonseca, S., Linstädter, J., Muianga, D., \&
Cascalheira, J. (2021). \emph{Online Education on African Archaeology
and Heritage: The ONLAAH Platform}. SAA 2021, Online.}

\bibitem[\citeproctext]{ref-regala_2021a}
\CSLLeftMargin{39. }%
\CSLRightInline{Regala, F., Bicho, N., Cascalheira, J., Gonçalves, C.,
\& André, L. (2021). \emph{Algarão da Figueira, Loulé, um novo sítio
arqueológico da Pré-História recente em contexto endocársico}. XI
Encontro de Arqueologia do Sudoeste Peninsular, Loulé.}

\bibitem[\citeproctext]{ref-belmiro_2020b}
\CSLLeftMargin{40. }%
\CSLRightInline{Belmiro, J., Cascalheira, J., \& Gonçalves, C. (2020).
\emph{Stone tool technology at the Cabeço da Amoreira shellmidden (Muge,
Portugal): A diachronic perspective}. MESO 2020, Toulouse, FR.}

\bibitem[\citeproctext]{ref-cascalheira_2019b}
\CSLLeftMargin{41. }%
\CSLRightInline{Cascalheira, J., Gonçalves, C., \& Bicho, N. (2019).
\emph{Assessing the spatial patterning of Middle Paleolithic human
settlement in Westernmost Iberia}. SAA 2019, Albuquerque, US.}

\bibitem[\citeproctext]{ref-horta_2019a}
\CSLLeftMargin{42. }%
\CSLRightInline{Horta, P., Cascalheira, J., \& Bicho, N. (2019).
\emph{Neanderthal ecological niche in Iberia's Southwestern edge: New
data from the Gruta da Companheira site}. SAA 2019, Albuquerque, US.}

\bibitem[\citeproctext]{ref-haws_2019c}
\CSLLeftMargin{43. }%
\CSLRightInline{Haws, J., Benedetti, M., Friedl, L., Bicho, N.,
Carvalho, M., Cascalheira, J., Ellis, G., Zinsious, B., Benedetti, I.,
\& Talamo, S. (2019). \emph{Modern human dispersal into western Iberia:
The Early Aurignacian of Lapa do Picareiro, Portugal}. ESHE 2019, Liège,
BE.}

\bibitem[\citeproctext]{ref-haws_2019a}
\CSLLeftMargin{44. }%
\CSLRightInline{Haws, J., Benedetti, M., Cascalheira, J., Bicho, N.,
Carvalho, M., Ellis, G., Zinsious, B., \& Friedl, L. (2019). \emph{Human
adaptive responses to abrupt climate change in western Iberia during MIS
3 and 2}. INQUA 2019, Dublin, IR.}

\bibitem[\citeproctext]{ref-haws_2019d}
\CSLLeftMargin{45. }%
\CSLRightInline{Haws, J., Benedetti, M., Pereira, T., Bicho, N.,
Cascalheira, J., \& Friedl, L. (2019). \emph{Human adaptive responses to
climate and environmental change during the Gravettian of Lapa do
Picareiro (Portugal)}. World of Gravettian Hunters 2019, Krakow, PL.}

\bibitem[\citeproctext]{ref-bicho_2019a}
\CSLLeftMargin{46. }%
\CSLRightInline{Bicho, N., Cascalheira, J., \& Haws, J. (2019).
\emph{Neandertais e Homens Anatomicamente Modernos no centro e sul de
Portugal}. 2º Colóquio Internacional - História das ideias e dos
conceitos em Arqueologia, Oeiras.}

\bibitem[\citeproctext]{ref-gomes_2019}
\CSLLeftMargin{47. }%
\CSLRightInline{Gomes, A., Moura, D., Bicho, N., Connor, S., Raja, M.,
Haws, J., Achimo, M., Zinsious, B., Skosey-Lalonde, E., Gonçalves, C.,
Oliveira, S., Matias, R., Fernandes, P., Costas, S., \& Cascalheira, J.
(2019). \emph{Southern Hemisphere climate changes studies: A
contribution for climate risk management and resilience}. 4th European
Climate Change Adaptation Conference, Lisbon, PT.}

\bibitem[\citeproctext]{ref-bicho_2019}
\CSLLeftMargin{48. }%
\CSLRightInline{Bicho, N., Cascalheira, J., Belmiro, J., \& Haws, J.
(2019). \emph{The Gravettian-Solutrean transition in Southwestern
Iberia}. The World of the Gravettian Hunters, Krakow, PL.}

\bibitem[\citeproctext]{ref-haws_2018a}
\CSLLeftMargin{49. }%
\CSLRightInline{Haws, J., Benedetti, M., Carvalho, M., Benedetti, I.,
Zinsious, B., Bicho, N., Cascalheira, J., \& Friedl, L. (2018).
\emph{Seeing short-term human occupations in the palimpsests of deeply
stratified caves: An example from Lapa do Picareiro (Portugal)}. UISPP
2018, Paris, FR.}

\bibitem[\citeproctext]{ref-haws_2018b}
\CSLLeftMargin{50. }%
\CSLRightInline{Haws, J., Benedetti, M., Carvalho, M., Bicho, N.,
Cascalheira, J., \& Friedl, L. (2018). \emph{Understanding human
adaptive responses to abrupt climate change in western Iberia during MIS
3 and 2 using multi-scale archaeological, paleoenvironmental, and
paleoclimate records.} UISPP 2018, Paris, FR.}

\bibitem[\citeproctext]{ref-gomes_2018}
\CSLLeftMargin{51. }%
\CSLRightInline{Gomes, A., Raja, M., Bicho, N., Haws, J., Moura, D.,
Achimo, M., Zinsious, B., Skosey-Lalonde, E., Gonçalves, C., Matias, R.,
Cascalheira, J., Connor, S., \& Oliveira, S. (2018).
\emph{Paleoenvironmental changes in Mozambique}. African Science and Art
Conference, Faro, PT.}

\bibitem[\citeproctext]{ref-horta_2017a}
\CSLLeftMargin{52. }%
\CSLRightInline{Horta, P., Cascalheira, J., \& Bicho, N. (2017).
\emph{Sobre a Definição e Interpretação das Tecnologias Líticas
Bipolares em Contextos Pré-Históricos}. II Congresso dos Arqueológos
Portugueses, Lisbon, PT.}

\bibitem[\citeproctext]{ref-cascalheira_2017b}
\CSLLeftMargin{53. }%
\CSLRightInline{Cascalheira, J., \& Bicho, N. (2017).
\emph{Territoriality and the organization of technology during the LGM
in Southern Iberia}. 3rd International Conference on the Solutrean,
Faro, PT.}

\bibitem[\citeproctext]{ref-abrunhosa_2017}
\CSLLeftMargin{54. }%
\CSLRightInline{Abrunhosa, A., Cascalheira, J., Pérez-González, A.,
Arsuaga, J. L., \& Baquedano, E. (2017). \emph{The use of digital mobile
technologies for geoarchaeological survey: The examples of the Pinilla
del Valle raw materials project}. 12th International Conference of
Archaeological Prospection, Bradford, UK.}

\bibitem[\citeproctext]{ref-bicho_2017}
\CSLLeftMargin{55. }%
\CSLRightInline{Bicho, N., \& Cascalheira, J. (2017). \emph{A critical
review of the meaning of Short-Term occupation in early Prehistory}. SAA
2017, Vancouver, CA.}

\bibitem[\citeproctext]{ref-gomes_2017a}
\CSLLeftMargin{56. }%
\CSLRightInline{Gomes, A., Skosey-Lalonde, E., Zinsious, B., Gonçalves,
C., Bicho, N., Raja, M., Cascalheira, J., \& Haws, J. (2017).
\emph{Environmental changes on the Mozambican coast, during the
Holocene}. IX Iberian Quaternary MeetingAfrican Science and Art
Conference, Faro, PT.}

\bibitem[\citeproctext]{ref-gomes_2017b}
\CSLLeftMargin{57. }%
\CSLRightInline{Gomes, A., Skosey-Lalonde, E., Zinsious, B., Gonçalves,
C., Bicho, N., Raja, M., Cascalheira, J., \& Haws, J. (2017).
\emph{Palaeoenvironmental reconstructions on the Mozambique coast as a
tool to understand human evolution: From modern analogues to borehole
interpretation}. European Geosciences Union General Assembly, Vienna,
AU.}

\bibitem[\citeproctext]{ref-bicho_2016b}
\CSLLeftMargin{58. }%
\CSLRightInline{Bicho, N., Marreiros, J., Cascalheira, J., \& Raja, M.
(2016). \emph{An Early Gravettian Point Cache from Vale Boi:
Implications for the Arrival of Anatomically Modern Humans to Southern
Iberia}. 81st Annual Meeting of the Society for American Archaeology,
Orlando.}

\bibitem[\citeproctext]{ref-bicho_2016}
\CSLLeftMargin{59. }%
\CSLRightInline{Bicho, N., Cascalheira, J., Haws, J., Gonçalves, C., \&
Raja, M. (2016). \emph{Middle Stone Age technologies in Mozambique:
Preliminary results}. ESHE 2016, Madrid, ES.}

\bibitem[\citeproctext]{ref-cascalheira_2016a}
\CSLLeftMargin{60. }%
\CSLRightInline{Cascalheira, J., \& Bicho, N. (2016). \emph{Changing
weapons in a mutable landscape: Exploring the relationship between Upper
Paleolithic Weaponry Variability and Drastic Environmental Changes in
Western Europe}. SAA 2017, Orlando, US.}

\bibitem[\citeproctext]{ref-haws_2016}
\CSLLeftMargin{61. }%
\CSLRightInline{Haws, J., Bicho, N., Gonçalves, C., Cascalheira, J.,
Pereira, T., Marreiros, J., Zinsious, B., Raja, M., Madime, O., \&
Benedetti, M. (2016). \emph{New data on Stone Age Archaeology South of
the rio Save, Mozambique}. 23 Biennal meeting of the Society of
Africanist Archaeologists. What past for Africa?, Toulouse, FR.}

\bibitem[\citeproctext]{ref-taylor_2016}
\CSLLeftMargin{62. }%
\CSLRightInline{Taylor, R., García-Rivero, D., Cascalheira, J., \&
Bicho, N. (2016). \emph{Technological diversity of the Early Neolithic
pottery of Muge (Portugal): The case study of Cabeço da Amoreira
(excavations from 2008 to 2014)}. Raw materials exploitation in
Prehistory: Sourcing, processing and distribution, Faro, PT.}

\bibitem[\citeproctext]{ref-gomes_2016}
\CSLLeftMargin{63. }%
\CSLRightInline{Gomes, A., Haws, J., Bicho, N., Gonçalves, C.,
Cascalheira, J., Zinsious, B., Skosey-Lalonde, E., Benedetti, I.,
Benedetti, M., Carvalho, M., \& Raja, M. (2016). \emph{The Stone Age in
Vilankulos: Investigation of the origin of Modern Human south of Save
River, Mozambique}. III Show of Science and Technology Projects, Tavira,
PT.}

\bibitem[\citeproctext]{ref-cascalheira_2015g}
\CSLLeftMargin{64. }%
\CSLRightInline{Cascalheira, J., Gonçalves, C., Umbelino, C., \& Bicho,
N. (2015). \emph{Rites, symbols and spatial organization of two human
burials at the Cabeço da Amoreira shellmidden (Muge, Portugal)}. MESO
2015, Belgrade, RS.}

\bibitem[\citeproctext]{ref-cascalheira_2015c}
\CSLLeftMargin{65. }%
\CSLRightInline{Cascalheira, J., Bicho, N., Gonçalves, C., Marreiros,
J., \& Paixão, E. (2015). \emph{Internal layout and functional
organization of Mesolithic shellmiddens: New insights from Cabeço da
Amoreira (Muge, Portugal)}. MESO 2015, Belgrade, RS.}

\bibitem[\citeproctext]{ref-cunha_2015}
\CSLLeftMargin{66. }%
\CSLRightInline{Cunha, C., Umbelino, C., Gonçalves, C., Figueiredo, O.,
Cascalheira, J., Marreiros, J., Pereira, T., Paixão, E., Monteiro, P.,
Évora, M., Dias, R., \& Bicho, N. (2015). \emph{Cultural change or
dental pathology? Dental analysis of two human burials from Cabeço da
Amoreira shell mound (Muge, Portugal)}. MESO 2015, Belgrade, RS.}

\bibitem[\citeproctext]{ref-aldeias_2015}
\CSLLeftMargin{67. }%
\CSLRightInline{Aldeias, V., Cascalheira, J., Marreiros, J., Pereira,
T., \& Bicho, N. (2015). \emph{How to make a Mesolithic Shell mound?
Microstratigraphic investigation of Cabeço da Amoreira (Muge,
Portugal)}. MESO 2015, Belgrade, RS.}

\bibitem[\citeproctext]{ref-monteiro_2015}
\CSLLeftMargin{68. }%
\CSLRightInline{Monteiro, P., Cascalheira, J., Marreiros, J., Pereira,
T., Umbelino, C., Dias, R., Gonçalves, C., Figueiredo, O., Évora, M.,
Paixão, E., Bicho, N., \& Zapata, L. (2015). \emph{A thousand years in
flames: A diachronical perspective on fuelwood use in Cabeço da Amoreira
(Muge shellmiddens, Portugal)}. MESO 2015, Belgrade, RS.}

\bibitem[\citeproctext]{ref-cascalheira_2015d}
\CSLLeftMargin{69. }%
\CSLRightInline{Cascalheira, J., Bicho, N., Marreiros, J., \& Pereira,
T. (2015). \emph{Climate-driven cultural change during the Upper
Paleolithic: A reassessment based on the Portuguese data.} ESHE 2015,
London, UK.}

\bibitem[\citeproctext]{ref-bicho_2015c}
\CSLLeftMargin{70. }%
\CSLRightInline{Bicho, N., Madiquida, H., Haws, J., Benedetti, M.,
Riel-Salvatore, J., Madime, O., Raja, M., Cascalheira, J., Gonçalves,
C., \& Pereira, T. (2015). \emph{A Idade da Pedra Média e as origens do
Homem Anatomicamente Moderno em Moçambique.} Seminário Internacional de
Arqueologia Africana: Arqueologia e Paisagem, Mação, PT.}

\bibitem[\citeproctext]{ref-cascalheira_2015b}
\CSLLeftMargin{71. }%
\CSLRightInline{Cascalheira, J., \& Bicho, N. (2015). \emph{Lithic
technological organization and social networks during the LGM in
Southwestern Iberia}. SAA 2015, San Francisco, US.}

\bibitem[\citeproctext]{ref-cascalheira_2015e}
\CSLLeftMargin{72. }%
\CSLRightInline{Cascalheira, J., Gonçalves, C., Aldeias, V., Benedetti,
M., Haws, J., Madime, O., Matos, D., Raja, M., Pereira, T., Zinsious,
B., \& Bicho, N. (2015). \emph{Stone Age occupations in Northern
Mozambique: New evidence from a survey project in the Lunho river valley
(Niassa)}. Paleoanthropological Society Annual Meeting, San Francisco,
US.}

\bibitem[\citeproctext]{ref-pereira_2015a}
\CSLLeftMargin{73. }%
\CSLRightInline{Pereira, T., Bicho, N., Cascalheira, J., Infantini, L.,
Marreiros, J., \& Paixão, E. (2015). \emph{Territory and abiotic economy
between 33 and 15,6 ka at Vale Boi (SW Portugal)}. UISPP 2015, Burgos,
ES.}

\bibitem[\citeproctext]{ref-cascalheira_2014a}
\CSLLeftMargin{74. }%
\CSLRightInline{Cascalheira, J., Paixão, E., \& Bicho, N. (2014).
\emph{On the border: The lithic assemblages from the Trench area of
Cabeço da Amoreira shellmidden (Central Portugal)}. SAA 2014, Austin,
US.}

\bibitem[\citeproctext]{ref-goncalves_2014a}
\CSLLeftMargin{75. }%
\CSLRightInline{Gonçalves, C., Cascalheira, J., Dias, R., Monteiro, P.,
Paixão, E., \& Bicho, N. (2014). \emph{Piece by piece: GIS spatial
analysis at Cabeço da Amoreira, a mesolithic shellmidden in Central
Portugal}. SAA 2014, Austin, US.}

\bibitem[\citeproctext]{ref-bicho_2014}
\CSLLeftMargin{76. }%
\CSLRightInline{Bicho, N., Cascalheira, J., Marreiros, J., \& Pereira,
T. (2014). \emph{Rapid cooling events, human resilience and
technological change: The case of the portuguese Upper Paleolithic}. SAA
2014, Austin, US.}

\bibitem[\citeproctext]{ref-pereira_2014}
\CSLLeftMargin{77. }%
\CSLRightInline{Pereira, T., Bicho, N., Cascalheira, J., Marreiros, J.,
\& Gonçalves, C. (2014). \emph{Testing the impact of coastal
environments in social inequality through lithic raw materials}. SAA
2014, Austin, US.}

\bibitem[\citeproctext]{ref-paixao_2014}
\CSLLeftMargin{78. }%
\CSLRightInline{Paixão, E., Cascalheira, J., Marreiros, J., Pereira, T.,
\& Bicho, N. (2014). \emph{Technological approaches to stone tool
production: The case of layer 2 of Mesolithic shellmidden of Cabeço da
Amoreira, Muge (Portugal)}. SAA 2014, Austin, US.}

\bibitem[\citeproctext]{ref-umbelino_health_2014}
\CSLLeftMargin{79. }%
\CSLRightInline{Umbelino, C., Gonçalves, C., Figueiredo, O.,
Cascalheira, J., Marreiros, J., Évora, M., Dias, R., Cunha, E., \&
Bicho, N. (2014). \emph{Health and diet in the Late Mesolithic: A
paleobiological perspective through the analysis of the human skeletons
retrieved from the Cabeço da Amoreira recent excavations}. SAA 2014,
Austin, US.}

\bibitem[\citeproctext]{ref-cascalheira_2013c}
\CSLLeftMargin{80. }%
\CSLRightInline{Cascalheira, J., Bicho, N., Marreiros, J., Pereira, T.,
\& Manne, T. (2013). \emph{Resilient adaptive patterns in the Upper
Paleolithic site of Vale Boi Southwestern Iberia)}. EAA 2013, Pilzen,
CZ.}

\bibitem[\citeproctext]{ref-bicho_2013e}
\CSLLeftMargin{81. }%
\CSLRightInline{Bicho, N., Umbelino, C., Gonçalves, C., Figueiredo, O.,
Pereira, T., Cascalheira, J., Marreiros, J., \& Price, T. D. (2013).
\emph{Human burials in the Mesolithic of Muge and the origins of social
differentiation: The case of Cabeço da Amoreira, Portugal}. Congress
Mesolithic burials -- Rites, symbols and social organisation of early
postglacial communities, Halle, DE.}

\bibitem[\citeproctext]{ref-bicho_2013c}
\CSLLeftMargin{82. }%
\CSLRightInline{Bicho, N., Pereira, T., Cascalheira, J., Marreiros, J.,
Gonçalves, C., Dias, R., Monteiro, P., \& Figueiredo, O. (2013).
\emph{The Mesolithic chronology of Cabeço da Amoreira (Muge, central
Portugal)}. Muge 150th, Muge, PT.}

\bibitem[\citeproctext]{ref-umbelino_life_2013}
\CSLLeftMargin{83. }%
\CSLRightInline{Umbelino, C., Gonçalves, C., Figueiredo, O., Pereira,
T., Cascalheira, J., Marreiros, J., Évora, M., Cunha, E., \& Bicho, N.
(2013). \emph{Life in the Muge shellmiddens: Inferences from the new
skeletons recovered from Cabeço da Amoreira}. Muge 150th, Muge, PT.}

\bibitem[\citeproctext]{ref-cascalheira_2012b}
\CSLLeftMargin{84. }%
\CSLRightInline{Cascalheira, J. (2012). \emph{The chronology of the
Iberic Solutrean: A critical review of the radiometric data}. Congreso
Internacional El Solutrense, Vélez Blanco, ES.}

\bibitem[\citeproctext]{ref-cascalheira_2012d}
\CSLLeftMargin{85. }%
\CSLRightInline{Cascalheira, J., Bicho, N., Cortés Sánchez, M., Évora,
M., Gibaja, J., Manne, T., Marreiros, J., Pereira, T., \& Tátá, F.
(2012). \emph{Vale Boi and the Solutrean in Southwestern Iberia}.
Congreso Internacional El Solutrense, Vélez Blanco, ES.}

\bibitem[\citeproctext]{ref-cascalheira_2012a}
\CSLLeftMargin{86. }%
\CSLRightInline{Cascalheira, J. (2012). \emph{Hunter-gatherer
ecodynamics and the impact of the H2 cold event in Central and Southern
Portugal}. SAA 2012, Memphis, US.}

\bibitem[\citeproctext]{ref-bicho_2012d}
\CSLLeftMargin{87. }%
\CSLRightInline{Bicho, N., Haws, J., Raposo, L., Brugal, J.-P.,
Cascalheira, J., Marreiros, J., \& Pereira, T. (2012). \emph{The end of
the Middle Paleolithic and the emergence of the anatomically modern
humans in Southwestern Iberia}. Paleoanthropological Society Annual
Meeting, Memphis, US.}

\bibitem[\citeproctext]{ref-bicho_2012e}
\CSLLeftMargin{88. }%
\CSLRightInline{Bicho, N., Manne, T., Marreiros, J., Cascalheira, J., \&
Pereira, T. (2012). \emph{The ecodynamics of the first modern humans in
Southwestern Iberia: The case of Vale Boi, Portugal}. SAA 2012, Memphis,
US.}

\bibitem[\citeproctext]{ref-monteiro_2012b}
\CSLLeftMargin{89. }%
\CSLRightInline{Monteiro, P., Cascalheira, J., Marreiros, J., Pereira,
T., \& Bicho, N. (2012). \emph{Fire as a component of Mesolithic
funerary rituals: The case of a burial in Cabeço da Amoreira (Muge,
Portugal)}. Where the Wild Things Are Conference: New research in
Palaeolithic and Mesolithic., Durham, UK.}

\bibitem[\citeproctext]{ref-monteiro_2012a}
\CSLLeftMargin{90. }%
\CSLRightInline{Monteiro, P., Cascalheira, J., Marreiros, J., Pereira,
T., \& Bicho, N. (2012). \emph{A Mesolithic landscape and woodland
resources: Wood charcoal analyses from Cabeço da Amoreira, Muge
shellmiddens (Santarém, Portugal)}. Association of Environmental
Archaeology, Spring Conference: New Trends on Environmental Archaeology,
Plymouth, UK.}

\bibitem[\citeproctext]{ref-cascalheira_2011a}
\CSLLeftMargin{91. }%
\CSLRightInline{Cascalheira, J. (2011). \emph{Variabilidade morfométrica
das pontas de pedúnculo central e aletas do Solutrense final peninsular:
Implicações estilísticas e territoriais}. IV Jornadas de Jovens em
Investigação Arqueológica, Faro, PT.}

\bibitem[\citeproctext]{ref-cascalheira_2011c}
\CSLLeftMargin{92. }%
\CSLRightInline{Cascalheira, J., \& Gonçalves, C. (2011). \emph{Análise
espacial e processos de formação do registo arqueológico no concheiro
mesolítico do Cabeço da Amoreira (Muge, Portugal)}. IV Jornadas de
Jovens em Investigação Arqueológica, Faro, PT.}

\bibitem[\citeproctext]{ref-marreiros_2011}
\CSLLeftMargin{93. }%
\CSLRightInline{Marreiros, J., Bicho, N., Cortés Sánchez, M., Gibaja,
J., Manne, T., Évora, M., Tátá, F., Cascalheira, J., \& Pereira, T.
(2011). \emph{New evidences from the Early Upper Paleolithic of
Southwestern Iberian Peninsula: The Gravettian of Vale Boi (Southern
Portugal)}. Reunión Gravetiense Cantábrico, Santander, ES.}

\bibitem[\citeproctext]{ref-dias_2011}
\CSLLeftMargin{94. }%
\CSLRightInline{Dias, R., Cascalheira, J., Gonçalves, C., Detry, C., \&
Bicho, N. (2011). \emph{Linking the bones to the stones: Preliminary
spatial analysis in the Cabeço da Amoreira shellmidden}. UKAS 2011,
Reading, UK.}

\bibitem[\citeproctext]{ref-cascalheira_2011}
\CSLLeftMargin{95. }%
\CSLRightInline{Cascalheira, J. (2011). \emph{A influência mediterrânica
nas redes sociais do Solutrense final peninsular}. Arqueologia em
Construção, UNIARQ, Lisbon, PT.}

\bibitem[\citeproctext]{ref-cascalheira_2011e}
\CSLLeftMargin{96. }%
\CSLRightInline{Cascalheira, J., \& Marreiros, J. (2011). \emph{Sobre a
descontrução de um concheiro: Metodologia de escavação e registo no
Cabeço da Amoreira}. Encontro sobre os concheiros de Muge: Analisando o
Passado, Lisbon, PT.}

\bibitem[\citeproctext]{ref-bicho_2011h}
\CSLLeftMargin{97. }%
\CSLRightInline{Bicho, N., Pereira, T., Gonçalves, C., Cascalheira, J.,
Marreiros, J., \& Dias, R. (2011). \emph{Os últimos
caçadores-recolectores do vale do Tejo: O caso de Muge}. International
Conference on the Prehistory of Wetlands, Setúbal, PT.}

\bibitem[\citeproctext]{ref-bicho_2010g}
\CSLLeftMargin{98. }%
\CSLRightInline{Bicho, N., Pereira, T., Umbelino, C., Jesus, L.,
Marreiros, J., \& Cascalheira, J. (2010). \emph{The construction of a
shellmidden: The case of Cabeço da Amoreira, Muge (Portugal)}. MESO
2010, Santander, ES.}

\bibitem[\citeproctext]{ref-marreiros_2010a}
\CSLLeftMargin{99. }%
\CSLRightInline{Marreiros, J., Jesus, L., Cascalheira, J., Pereira, T.,
Gibaja, J., \& Bicho, N. (2010). \emph{{``Shell we move?''} New
technological approach to Mesolithic settlement patterns at Muge
(Portuguese Estremadura)}. MESO 2010, Santander, ES.}

\bibitem[\citeproctext]{ref-marreiros_2010}
\CSLLeftMargin{100. }%
\CSLRightInline{Marreiros, J., \& Cascalheira, J. (2010). \emph{Veni,
vidi, vici! Lo mejor de ambos mundos. Modelo teórico sobre la ocupación
del Hombre moderno en el extremo suroeste de la Península Ibérica}. III
Jornadas de Jóvenes en Investigación Arqueológica, Barcelona, ES.}

\bibitem[\citeproctext]{ref-cascalheira_2010a}
\CSLLeftMargin{101. }%
\CSLRightInline{Cascalheira, J. (2010). \emph{Territoriality and social
networks on the Upper Solutrean of Southern Iberia: A review of facts
and perspectives for the future}. III Jornadas de Jóvenes en
Investigación Arqueológica, Barcelona, ES.}

\bibitem[\citeproctext]{ref-bicho_2010b}
\CSLLeftMargin{102. }%
\CSLRightInline{Bicho, N., Manne, T., Marreiros, J., Cascalheira, J.,
Mendonça, C., Gibaja, J., Évora, M., \& Pereira, T. (2010). \emph{O
Paleolítico superior do Sudoeste da Península Ibérica: O caso do
Algarve}. Jornadas Internacionales sobre el Paleolítico superior
peninsular, Barcelona, ES.}

\bibitem[\citeproctext]{ref-bicho_2009e}
\CSLLeftMargin{103. }%
\CSLRightInline{Bicho, N., Pereira, T., Cascalheira, J., Marreiros, J.,
\& Pereira, V. (2009). \emph{Cabeço da Amoreira, Muge: Resultados dos
trabalhos de 2008 e 2009}.}

\bibitem[\citeproctext]{ref-bicho_2009c}
\CSLLeftMargin{104. }%
\CSLRightInline{Bicho, N., Cascalheira, J., Cortés Sánchez, M., Gibaja,
J., Évora, M., Manne, T., Marreiros, J., Mendonça, C., Pereira, T., \&
Tátá, F. (2009). \emph{Identidade e adaptação: A ocupação humana durante
o Plistocénico final no Algarve Ocidental}. VII Reunião do Quaternário
Ibérico, Faro, PT.}

\bibitem[\citeproctext]{ref-bicho_2009d}
\CSLLeftMargin{105. }%
\CSLRightInline{Bicho, N., Manne, T., Marreiros, J., Tátá, F., Gibaja,
J., Évora, M., \& Cascalheira, J. (2009). \emph{The first modern humans
in Southwestern Portugal: The Gravettian from Vale Boi}. CALPE
conference 2009 -- Human evolution, 150 years after Darwin, Gibraltar,
UK.}

\bibitem[\citeproctext]{ref-cascalheira_2009a}
\CSLLeftMargin{106. }%
\CSLRightInline{Cascalheira, J. (2009). \emph{Estilo, identidade e
adaptação: O abrigo de Vale Boi no quadro do Solutrense peninsular}.
Associação dos arqueólogos portugueses, Lisbon, PT.}

\bibitem[\citeproctext]{ref-bicho_2008}
\CSLLeftMargin{107. }%
\CSLRightInline{Bicho, N., Cascalheira, J., Gibaja, J., \& Marreiros, J.
(2008). \emph{Caracterización de la industria gravetiense y solutrense
de Vale Boi (Algarve, Portugal)}. IV Encontro de Arqueologia do Sudoeste
Peninsular, Aracena, ES.}

\bibitem[\citeproctext]{ref-gibaja_2008}
\CSLLeftMargin{108. }%
\CSLRightInline{Gibaja, J., Marreiros, J., Cascalheira, J., Palomo, A.,
Carvalho, A. F., \& Rojo, M. (2008). \emph{Análisis traceológico del
utillaje lítico dcumentado en el asentamiento neolítico de Zafrín (Islas
Chafarinas). Configuración de un programa experimental dirigido al
reconocimiento del uso de los perforadores}. II Congreso Internacional
de Arqueologia Experimental, Ronda, ES.}

\bibitem[\citeproctext]{ref-bicho_2008b}
\CSLLeftMargin{109. }%
\CSLRightInline{Bicho, N., Cascalheira, J., \& Marreiros, J. (2008).
\emph{On the (l)edge: The case of Vale Boi rockshelter (Algarve,
Southern Portugal)}. EAA 2008, Malta, MT.}

\bibitem[\citeproctext]{ref-cascalheira_2008}
\CSLLeftMargin{110. }%
\CSLRightInline{Cascalheira, J. (2008). \emph{Tecnologia lítica
solutrense del abrigo de Vale Boi (Algarve, Portugal): Resultados
preliminares}. I Jornadas de Jóvenes en Investigación Arqueológica,
Madrid, ES.}

\bibitem[\citeproctext]{ref-bicho_2008a}
\CSLLeftMargin{111. }%
\CSLRightInline{Bicho, N., Cascalheira, J., \& Marreiros, J. (2008).
\emph{As intervenções arqueológicas de 2006 e 2007 no sítio Paleolítico
de Vale Boi}. 5º Encontro de Arqueologia do Algarve, Silves, PT.}

\end{CSLReferences}

\subsection{\texorpdfstring{\ul{Posters}}{Posters}}\label{posters}

\phantomsection\label{refs-76cb30b9c6f851aec58215c378661ba7}
\begin{CSLReferences}{0}{0}
\bibitem[\citeproctext]{ref-cascalheira_2024a}
\CSLLeftMargin{1. }%
\CSLRightInline{Cascalheira, J. (2024). \emph{FINISTERRA--Population
Trajectories and Cultural Dynamics of Late Neanderthals in Far Western
Eurasia}. SAA 2024, New Orleans (US).}

\bibitem[\citeproctext]{ref-cobo-sanchez_2024}
\CSLLeftMargin{2. }%
\CSLRightInline{Cobo-Sánchez, L., \& Cascalheira, J. (2024).
\emph{Neanderthal and Carnivore Interplay at Escoural Cave: Preliminary
Evidence from the Archaeofaunal and Spatial Analysis of Two New Test
Pit}. SAA 2024, New Orleans (US).}

\bibitem[\citeproctext]{ref-belmiro_2023a}
\CSLLeftMargin{3. }%
\CSLRightInline{Belmiro, J., Terradas, X., Bicho, N., \& Cascalheira, J.
(2023). \emph{New insights into the chert sources of southern Portugal:
A macroscopic and petrographic approach}. Paleoanthropological Meetings
2023, Portland, USA.}

\bibitem[\citeproctext]{ref-cascalheira_2023}
\CSLLeftMargin{4. }%
\CSLRightInline{Cascalheira, J., Barbieri, A., Carvalho, M., Galfi, J.,
Gomes, A., Gonçalves, C., Horta, P., Maio, D., Matias, R., Medialdea,
A., Richard, M., \& Del Val Blanco, M. (2023). \emph{Gruta do Escoural:
New data on the Middle Paleolithic occupation of southwestern Iberia}.
Paleoanthropological Meetings 2023, Portlan, USA.}

\bibitem[\citeproctext]{ref-goncalves_2023a}
\CSLLeftMargin{5. }%
\CSLRightInline{Gonçalves, C., Maio, D., \& Cascalheira, J. (2023).
\emph{Spatial patterns and trends of Middle Paleolithic human settlement
systems in southern Iberia}. Paleoanthropological Meetings 2023,
Portland, USA.}

\bibitem[\citeproctext]{ref-carvalho_2023}
\CSLLeftMargin{6. }%
\CSLRightInline{Carvalho, M., Benedetti, M., Cascalheira, J., Friedl,
L., \& Haws, J. (2023). \emph{Neanderthals and early modern humans in
western Iberia: Taphonomy and diet at Lapa do Picareiro (central
Portugal)}. ESHE 2023, Aarhus, DK.}

\bibitem[\citeproctext]{ref-horta_2021a}
\CSLLeftMargin{7. }%
\CSLRightInline{Horta, P., Cascalheira, J., \& Bicho, N. (2021).
\emph{Lithic Adaptive Strategies of Early Modern Humans in Southwestern
Iberia: New Data from Vale Boi's Layer 7 and 8}. SAA 2021, Online.}

\bibitem[\citeproctext]{ref-barbieri_2021c}
\CSLLeftMargin{8. }%
\CSLRightInline{Barbieri, A., Carvalho, M., Horta, P., Matias, R., Maio,
D., Aldeias, V., Bicho, N., \& Cascalheira, J. (2021). \emph{Tracing the
primary context of Neanderthal occupations within a complex karst system
-- Preliminary geoarchaeological results from Escoural (Alentejo,
Southern Portugal)}. DIG 2021, Online.}

\bibitem[\citeproctext]{ref-cascalheira_2021a}
\CSLLeftMargin{9. }%
\CSLRightInline{Cascalheira, J., Barbieri, A., Carvalho, M., Galfi, J.,
Gonçalves, C., Horta, P., Matias, R., \& Maio, D. (2021). \emph{Middle
Paleolithic occupations in the Escoural Cave (southern Portugal):
Preliminary geoarchaeological results}. ESHE Meetings 2021, Online.}

\bibitem[\citeproctext]{ref-skosey-lalonde_2020}
\CSLLeftMargin{10. }%
\CSLRightInline{Skosey-Lalonde, E., Gomes, A., Martins, M. J., Connor,
S., Raja, M., Zinsious, B., Matias, R., Mauelele, R., Haws, J., Moura,
D., Jasso, A., Hartman, G., Gonçalves, C., Cascalheira, J., Oliveira,
S., Fernandes, P., Costas, S., \& Bicho, N. (2020). \emph{Biogeochemical
analysis of newly dated lacustrine cores: A first look at Quaternary
paleoenvironment in coastal Mozambique}. EGU Genneral Assembly 2020,
Online.}

\bibitem[\citeproctext]{ref-gomes_2020}
\CSLLeftMargin{11. }%
\CSLRightInline{Gomes, A., Connor, S., Martins, M. J., Zinsious, B.,
Gonçalves, C., Moura, D., Skosey-Lalonde, E., Cascalheira, J., Haws, J.,
Nhanombe, J., Raja, M., Fernandes, P., Mauelele, R., Matias, R.,
Oliveira, S., Costas, S., \& Bicho, N. (2020). \emph{Modern environment
characterization of interdunal lakes in Inhambane province (SE
Mozambique) as an analogue to understand past environmental changes}.
EGU Genneral Assembly 2020, Online.}

\bibitem[\citeproctext]{ref-belmiro_2020d}
\CSLLeftMargin{12. }%
\CSLRightInline{Belmiro, J., Cascalheira, J., \& Gonçalves, C. (2020).
\emph{Uma perspectiva diacrónica da ocupação do concheiro do Cabeço da
Amoreira (Muge, Portugal) a partir da tecnologia lítica}. III Congresso
da Associação dos Arqueólogos Portugueses, Porto, PT.}

\bibitem[\citeproctext]{ref-cascalheira_2020c}
\CSLLeftMargin{13. }%
\CSLRightInline{Cascalheira, J., Belmiro, J., \& Gonçalves, C. (2020).
\emph{Variability of microliths morphology at the Cabeço da Amoreira
shellmound: An approach using Geometric Morphometrics}. MESO 2020,
Toulouse, FR.}

\bibitem[\citeproctext]{ref-goncalves_2020a}
\CSLLeftMargin{14. }%
\CSLRightInline{Gonçalves, C., Umbelino, C., Gomes, A., Gonçalves, C.,
Costa, C., Belmiro, J., Cascalheira, J., Cardoso, J. L., Rodrigues, J.,
André, L., Évora, M., Figueiredo, M., Santos, M., Zacarias, M., Bicho,
N., Monteiro, P., Godinho, R. M., Matias, R., \& Aldeias, V. (2020).
\emph{Muge Portal: A new digital platform for the last hunter-gatherers
of the Tagus Valley, Portugal.~10th}. MESO 2020, Toulouse, FR.}

\bibitem[\citeproctext]{ref-goncalves_2019}
\CSLLeftMargin{15. }%
\CSLRightInline{Gonçalves, C., Umbelino, C., \& Cascalheira, J. (2019).
\emph{Muge Portal: A new digital platform for the last hunter-gatherers
of the Tagus valley, Portugal}. SAA 2019, Albuquerque, US.}

\bibitem[\citeproctext]{ref-belmiro_2019a}
\CSLLeftMargin{16. }%
\CSLRightInline{Belmiro, J., Cascalheira, J., \& Gonçalves, C. (2019).
\emph{A geometric morphometrics approach to test the microlith
variability at Cabeço da Amoreira shellmidden (Muge, Portugal)}. SAA
2019, Albuquerque, US.}

\bibitem[\citeproctext]{ref-raja_2019}
\CSLLeftMargin{17. }%
\CSLRightInline{Raja, M., Bicho, N., Haws, J., Cascalheira, J.,
Skosey-Lalonde, E., Zinsious, B., Benedetti, I., Benedetti, M.,
Gonçalves, C., Achimo, M., Carvalho, M., \& Gomes, A. (2019).
\emph{Deciphering sediments in archaeological context: Inferences on the
palaeoenvironmental changes and the site formation processes in the
Middle and Late Stone Age, Txina-Txina, Massingir (Mozambique)}.
Paleoanthropological Meetings 2019, Albuquerque, US.}

\bibitem[\citeproctext]{ref-belmiro_2019}
\CSLLeftMargin{18. }%
\CSLRightInline{Belmiro, J., Cascalheira, J., Bicho, N., \& Haws, J.
(2019). \emph{The Gravettian-Solutrean transition in western Iberia: New
data from the sites of Vale Boi and Lapa do Picareiro (Portugal)}.
Paleoanthropological Meetings 2019, Albuquerque, US.}

\bibitem[\citeproctext]{ref-haws_2019b}
\CSLLeftMargin{19. }%
\CSLRightInline{Haws, J., Benedetti, M., Friedl, L., Bicho, N.,
Cascalheira, J., Carvalho, M., Ellis, G., Zinsious, B., \& Talamo, S.
(2019). \emph{Earlier than thought: The Early Aurignacian technocomplex
at Lapa do Picareiro (Portugal) and the arrival of modern humans in
southwestern Iberia 40,000 years ago}. Paleoanthropological Meetings
2019, Albuquerque, US.}

\bibitem[\citeproctext]{ref-horta_2019b}
\CSLLeftMargin{20. }%
\CSLRightInline{Horta, P., Cascalheira, J., Raja, M., \& Bicho, N.
(2019). \emph{Lithic bipolar reduction strategies in the Late Stone Age
site of Txina Txina, Mozambique}. ESHE 2019, Liège, BE.}

\bibitem[\citeproctext]{ref-cascalheira_2018f}
\CSLLeftMargin{21. }%
\CSLRightInline{Cascalheira, J., Cardoso, J. L., \& Martins, F. (2018).
\emph{A geometric morphometric approach to predict the chronological
attribution of bifacial foliate technology at Olival do Arneiro (central
Portugal)}. ESHE 2018, Faro, PT.}

\bibitem[\citeproctext]{ref-belmiro_2018}
\CSLLeftMargin{22. }%
\CSLRightInline{Belmiro, J., Cascalheira, J., \& Bicho, N. (2018).
\emph{At the threshold of the Last Glacial Maximum in southwestern
Iberia: New evidence from the site of Vale Boi (Portugal)}. ESHE 2018,
Faro, PT.}

\bibitem[\citeproctext]{ref-horta_2018}
\CSLLeftMargin{23. }%
\CSLRightInline{Horta, P., Cascalheira, J., \& Bicho, N. (2018).
\emph{Lithic bipolar technology through space and time}. ESHE 2018,
Faro, PT.}

\bibitem[\citeproctext]{ref-haws_2018c}
\CSLLeftMargin{24. }%
\CSLRightInline{Haws, J., Benedetti, M., Friedl, L., Bicho, N.,
Cascalheira, J., \& Carvalho, M. (2018). \emph{The Middle-Upper
Paleolithic Transition in Southern Iberia: New Data from Lapa do
Picareiro, Portugal}. ESHE 2018, Faro, PT.}

\bibitem[\citeproctext]{ref-haws_2018d}
\CSLLeftMargin{25. }%
\CSLRightInline{Haws, J., Bicho, N., Cascalheira, J., Gonçalves, C.,
Raja, M., Benedetti, M., Gomes, A., Carvalho, M., Benedetti, I.,
Zinsious, B., \& Skosey-Lalonde, E. (2018). \emph{Later Stone Age
archaeology in the Limpopo river basin: New evidence from the site of
Txina Txina (Massingir, southern Mozambique)}. UISPP 2018, Paris, FR.}

\bibitem[\citeproctext]{ref-benedetti_2018}
\CSLLeftMargin{26. }%
\CSLRightInline{Benedetti, I., Haws, J., Bicho, N., Raja, M.,
Cascalheira, J., Zinsious, B., \& Benedetti, M. (2018). \emph{Late
Holocene Shell Midden and Human Remains from Praia da Rocha,
Moçambique}. UISPP 2018, Paris, FR.}

\bibitem[\citeproctext]{ref-haws_2018}
\CSLLeftMargin{27. }%
\CSLRightInline{Haws, J., Benedetti, M., Carvalho, M., Bicho, N.,
Cascalheira, J., \& Friedl, L. (2018). \emph{Human adaptive responses to
abrupt climate change during the Late Pleistocene}. Paleoanthropological
Society Annual Meeting, Austin, US.}

\bibitem[\citeproctext]{ref-belmiro_2017a}
\CSLLeftMargin{28. }%
\CSLRightInline{Belmiro, J., Cascalheira, J., \& Bicho, N. (2017).
\emph{O início do último máximo glacial no Sul de Portugal: Novos dados
a partir do sítio arqueológico de Vale Boi}. II Congresso dos
Arqueológos Portugueses, Lisbon, PT.}

\bibitem[\citeproctext]{ref-varela_2017a}
\CSLLeftMargin{29. }%
\CSLRightInline{Varela, J., Bicho, N., Gonçalves, C., \& Cascalheira, J.
(2017). \emph{Análise preliminar dos padrões de localização das grutas
com arqueologia do centro e Sul de Portugal.} II Congresso dos
Arqueológos Portugueses, Lisbon, PT.}

\bibitem[\citeproctext]{ref-belmiro_2017b}
\CSLLeftMargin{30. }%
\CSLRightInline{Belmiro, J., Cascalheira, J., \& Bicho, N. (2017).
\emph{The beginning of the Last Glacial Maximum in southern Portugal:
New data from the archaeological site of Vale Boi}. 3rd International
Conference on the Solutrean, Faro, PT.}

\bibitem[\citeproctext]{ref-simonvallejo_2017}
\CSLLeftMargin{31. }%
\CSLRightInline{Simón Vallejo, M. D., Bicho, N., Cortés Sánchez, M.,
Parrila-Giraldez, R., \& Cascalheira, J. (2017). \emph{Nuevos datos de
arte mueble del yacimiento de Vale Boi (Algarve, Portugal)}. 3rd
International Conference on the Solutrean, Faro, PT.}

\bibitem[\citeproctext]{ref-cascalheira_2017f}
\CSLLeftMargin{32. }%
\CSLRightInline{Cascalheira, J., Gonçalves, C., Bicho, N., Gomes, A.,
Raja, M., \& Haws, J. (2017). \emph{Systematic sampling survey for Stone
Age sites in the Limpopo basin, SW Mozambique}. ESHE 2017, Leiden, NL.}

\bibitem[\citeproctext]{ref-bicho_2017d}
\CSLLeftMargin{33. }%
\CSLRightInline{Bicho, N., Cascalheira, J., Haws, J., Gonçalves, C.,
Raja, M., André, L., Benedetti, M., Carvalho, M., \& Zinsious, B.
(2017). \emph{Txina, Txina, a new LSA site from the Limpopo River
Valley, Mozambique}. ESHE 2017, Leiden, NL.}

\bibitem[\citeproctext]{ref-goncalves_2017a}
\CSLLeftMargin{34. }%
\CSLRightInline{Gonçalves, C., Cascalheira, J., Haws, J., Raja, M., \&
Bicho, N. (2017). \emph{Reconstructing Stone Age Settlement Patterns in
the Elephant river valley, Mozambique}. ESHE 2017, Leiden, NL.}

\bibitem[\citeproctext]{ref-horta_2017b}
\CSLLeftMargin{35. }%
\CSLRightInline{Horta, P., Cascalheira, J., Bicho, N., \& Tátá, F.
(2017). \emph{The Neanderthal occupations in Southwestern Iberia:
Preliminary data from the Gruta da Companheira site}. ESHE 2017, Leiden,
NL.}

\bibitem[\citeproctext]{ref-cascalheira_2017e}
\CSLLeftMargin{36. }%
\CSLRightInline{Cascalheira, J., Gonçalves, C., \& Bicho, N. (2017).
\emph{An Android-based system for archaeological survey and on-site
stone tool analysis}. SAA 2017, Vancouver, CA.}

\bibitem[\citeproctext]{ref-goncalves_2017}
\CSLLeftMargin{37. }%
\CSLRightInline{Gonçalves, C., Cascalheira, J., Haws, J., Raja, M., \&
Bicho, N. (2017). \emph{GIS and Archaeological survey data for the
reconstruction of stone age settlement patterns in the Elephant River
valley, Mozambique}. SAA 2017, Vancouver, CA.}

\bibitem[\citeproctext]{ref-cascalheira_2017d}
\CSLLeftMargin{38. }%
\CSLRightInline{Cascalheira, J., Bicho, N., Haws, J., Gonçalves, C., \&
Raja, M. (2017). \emph{Technological variability in the Middle Stone Age
of Mozambique: Preliminary results}. Paleoanthropological Society Annual
Meeting, Vancouver, CA.}

\bibitem[\citeproctext]{ref-goncalves_2017b}
\CSLLeftMargin{39. }%
\CSLRightInline{Gonçalves, C., Gomes, A., Cascalheira, J., Raja, M.,
Bicho, N., \& Haws, J. (2017). \emph{Applying systematic sampling survey
to evaluate Stone Age settlement in the Elephant River}.
Paleoanthropological Society Annual Meeting, Vancouver, CA.}

\bibitem[\citeproctext]{ref-bicho_2017c}
\CSLLeftMargin{40. }%
\CSLRightInline{Bicho, N., Cascalheira, J., Haws, J., Gonçalves, C.,
Raja, M., André, L., Benedetti, M., Gomes, A., Carvalho, M., \&
Zinsious, B. (2017). \emph{Txina Txina: A Later Stone Age site from the
Limpopo basin in southern Mozambique}. Paleoanthropological Society
Annual Meeting, Vancouver, CA.}

\bibitem[\citeproctext]{ref-gomes_2017}
\CSLLeftMargin{41. }%
\CSLRightInline{Gomes, A., Skosey-Lalonde, E., Zinsious, B., Gonçalves,
C., Bicho, N., Raja, M., Cascalheira, J., \& Haws, J. (2017).
\emph{Diatoms from Mozambique: A tool for palaeoenvironmental
reconstructions and to understand human evolution}. 5th Open Science
Meeting {``Global Challenges for our Common Future: A Paleoscience
perspective,''} Zaragoza, ES.}

\bibitem[\citeproctext]{ref-haws_2016a}
\CSLLeftMargin{42. }%
\CSLRightInline{Haws, J., Bicho, N., Gonçalves, C., Cascalheira, J.,
Benedetti, M., Pereira, T., Marreiros, J., Zinsious, B., Carvalho, M.,
\& Raja, M. (2016). \emph{Stone Age archaeology and Quaternary
landscapes in Southern Mozambique}. The Geological Society of America
Annual Meeting, Denver, US.}

\bibitem[\citeproctext]{ref-goncalves_2016}
\CSLLeftMargin{43. }%
\CSLRightInline{Gonçalves, C., Cascalheira, J., Haws, J., Zinsious, B.,
Raja, M., \& Bicho, N. (2016). \emph{Stone Age settlement patterns in
the Lunho valley (Niassa, Mozambique): GIS preliminary results}. ESHE
2016, Madrid, ES.}

\bibitem[\citeproctext]{ref-paixao_2016}
\CSLLeftMargin{44. }%
\CSLRightInline{Paixão, E., Marreiros, J., Gibaja, J., Pereira, T.,
Cascalheira, J., \& Bicho, N. (2016). \emph{Testing raw material
suitability for Ground stones: Experimental program in quartzite and
greywacke}. Raw materials exploitation in Prehistory: Sourcing,
processing and distribution, Faro, PT.}

\bibitem[\citeproctext]{ref-horta_2015}
\CSLLeftMargin{45. }%
\CSLRightInline{Horta, P., Cascalheira, J., \& Bicho, N. (2015).
\emph{Expedient tools for intensive practices: The bipolar lithic
implements from the Upper Paleolithic site of Vale Boi (Southwestern
Iberia)}. ESHE 2015, London.}

\bibitem[\citeproctext]{ref-cascalheira_2015a}
\CSLLeftMargin{46. }%
\CSLRightInline{Cascalheira, J., \& Bicho, N. (2015). \emph{Abrupt
climate change and the variability of technological patterns during the
Late Pleistocene in Western Iberia}. AAA 2015, Denver, US.}

\bibitem[\citeproctext]{ref-bicho_2015b}
\CSLLeftMargin{47. }%
\CSLRightInline{Bicho, N., Haws, J., Cascalheira, J., Gonçalves, C.,
Raja, M., Madime, O., Aldeias, V., Benedetti, M., Pereira, T.,
Marreiros, J., \& Riel-Salvatore, J. (2015). \emph{New survey results
from Middle and Late Stone Age in Mozambique}. AAA 2015, Denver, US.}

\bibitem[\citeproctext]{ref-goncalves_2015a}
\CSLLeftMargin{48. }%
\CSLRightInline{Gonçalves, C., Aldeias, V., Benedetti, M., Haws, J.,
Madime, O., Raja, M., Pereira, T., Zinsious, B., \& Bicho, N. (2015).
\emph{GIS and archaeological survey data for the reconstruction of Stone
Age settlement patterns in the Lunho valley (Niassa, Mozambique):
Preliminary results}. AAA 2015, Denver, US.}

\bibitem[\citeproctext]{ref-goncalves_2015b}
\CSLLeftMargin{49. }%
\CSLRightInline{Gonçalves, C., Cascalheira, J., \& Bicho, N. (2015).
\emph{GIS-based visibility studies in the Muge valley shellmiddens:
Implications for spatial and social organization}. MESO 2015, Belgrade,
RS.}

\bibitem[\citeproctext]{ref-paixao_2015}
\CSLLeftMargin{50. }%
\CSLRightInline{Paixão, E., Marreiros, J., Gibaja, J., Pereira, T.,
Cascalheira, J., \& Bicho, N. (2015). \emph{Living and hunting during
the Mesolithic in the Cabeço da Amoreira Shellmidden (Muge, Portugal):
Preliminary lithic use-wear analysis}. MESO 2015, Belgrade, RS.}

\bibitem[\citeproctext]{ref-horta_2015a}
\CSLLeftMargin{51. }%
\CSLRightInline{Horta, P., Cascalheira, J., Marreiros, J., \& Bicho, N.
(2015). \emph{Preliminary technological comparison between chert and
quartz splintered pieces from the Upper Paleolithic of Vale Boi
(Southwestern Iberia)}. On the Rocks 10th International Symposium on
Knappable Materials, Barcelona, ES.}

\bibitem[\citeproctext]{ref-cascalheira_2015f}
\CSLLeftMargin{52. }%
\CSLRightInline{Cascalheira, J., Gonçalves, C., Aldeias, V., Benedetti,
M., Haws, J., Madime, O., Raja, M., Pereira, T., \& Bicho, N. (2015).
\emph{Surface lithic scatters in the Lunho river valley (Niassa,
Mozambique): The case of Ncuala}. Seminário Internacional de Arqueologia
Africana: Arqueologia e Paisagem, Mação, PT.}

\bibitem[\citeproctext]{ref-goncalves_2015}
\CSLLeftMargin{53. }%
\CSLRightInline{Gonçalves, C., Aldeias, V., Benedetti, M., Cascalheira,
J., Haws, J., Madime, O., Raja, M., Pereira, T., \& Bicho, N. (2015).
\emph{Chicaza: A new rockshelter in the Chitete river valley (Niassa,
Mozambique)}. Seminário Internacional de Arqueologia Africana:
Arqueologia e Paisagem, Mação, PT.}

\bibitem[\citeproctext]{ref-goncalves_2015c}
\CSLLeftMargin{54. }%
\CSLRightInline{Gonçalves, C., Cascalheira, J., Raja, M., Madime, O., \&
Bicho, N. (2015). \emph{Mapping the Stone Age in Mozambique: Preliminary
Results}. SAA 2015, San Francisco, US.}

\bibitem[\citeproctext]{ref-bicho_2015}
\CSLLeftMargin{55. }%
\CSLRightInline{Bicho, N., Aldeias, V., Benedetti, M., Cascalheira, J.,
Gonçalves, C., Haws, J., Madime, O., Pereira, T., Raja, M., \& Zinsious,
B. (2015). \emph{Chicaza rockshelter (Niassa, Mozambique): A preliminary
report on stratigraphy and chronology}. Paleoanthropological Society
Annual Meeting, San Francisco, US.}

\bibitem[\citeproctext]{ref-paixao_2013}
\CSLLeftMargin{56. }%
\CSLRightInline{Paixão, E., Marreiros, J., Cascalheira, J., Pereira, T.,
Gibaja, J., \& Bicho, N. (2013). \emph{Technological approaches on chert
stone tool production in the Mesolithic of Muge (Portuguese
Estremadura)}. EAA 2013, Pilzen, CZ.}

\bibitem[\citeproctext]{ref-goncalves_2013}
\CSLLeftMargin{57. }%
\CSLRightInline{Gonçalves, C., Cascalheira, J., \& Bicho, N. (2013).
\emph{GIS predictive modeling for the discovery of new Mesolithic sites
in Central Portugal}. EAA 2013, Pilzen, CZ.}

\bibitem[\citeproctext]{ref-cascalheira_2013d}
\CSLLeftMargin{58. }%
\CSLRightInline{Cascalheira, J., Marreiros, J., Paixão, E., Pereira, T.,
\& Bicho, N. (2013). \emph{Techno-typological analysis of the lithic
materials from the Trench area of Cabeço da Amoreira (Muge, Central
Portugal)}. Muge 150th, Muge, PT.}

\bibitem[\citeproctext]{ref-conyers_2013}
\CSLLeftMargin{59. }%
\CSLRightInline{Conyers, L., Bicho, N., Daniels, M., Elliot, K., Castro,
G., Cascalheira, J., \& Gonçalves, C. (2013). \emph{Ground-penetrating
radar mapping at the Mesolitic Muge Shell Mound, Portugal}. Muge 150th,
Muge, PT.}

\bibitem[\citeproctext]{ref-schmidt_2012a}
\CSLLeftMargin{60. }%
\CSLRightInline{Schmidt, I., \& Cascalheira, J. (2012). \emph{What's the
difference? Variability among Solutrean {``Parpallo points''} from
Iberia}. Paleoanthropological Society Annual Meeting, Memphis, US.}

\bibitem[\citeproctext]{ref-goncalves_2012a}
\CSLLeftMargin{61. }%
\CSLRightInline{Gonçalves, C., Cascalheira, J., \& Bicho, N. (2012).
\emph{Preliminary spatial density analysis in the Upper Paleolithic
rockshelter of Vale Boi (Southern Portugal)}. Paleoanthropological
Society Annual Meeting, Memphis, US.}

\bibitem[\citeproctext]{ref-goncalves_2012b}
\CSLLeftMargin{62. }%
\CSLRightInline{Gonçalves, C., Cascalheira, J., \& Bicho, N. (2012).
\emph{Visibility studies on the Mesolithic of Muge valley (central
Portugal)}. SAA 2012, Memphis, US.}

\bibitem[\citeproctext]{ref-marreiros_2012a}
\CSLLeftMargin{63. }%
\CSLRightInline{Marreiros, J., Bicho, N., Gibaja, J., Cascalheira, J.,
\& Pereira, T. (2012). \emph{The Vincentine points from the Early
Gravettian of Vale Boi (Portugal): Technological and use-wear analysis}.
Paleoanthropological Society Annual Meeting, Memphis, US.}

\bibitem[\citeproctext]{ref-pereira_2012}
\CSLLeftMargin{64. }%
\CSLRightInline{Pereira, T., Marreiros, J., Cascalheira, J., Gibaja, J.,
Coelho, C., Bicho, N., \& Haws, J. (2012). \emph{Differences in raw
material management by Neandertals and modern humans in Southwest
Iberia}. Paleoanthropological Society Annual Meeting, Memphis, US.}

\bibitem[\citeproctext]{ref-goncalves_2012}
\CSLLeftMargin{65. }%
\CSLRightInline{Gonçalves, C., Cascalheira, J., Amorim, A., \& Bicho, N.
(2012). \emph{No more pencils, no more field books\ldots Archaeological
drawing from total station data and digital photography}. CAA 2012,
Southampton, UK.}

\bibitem[\citeproctext]{ref-cascalheira_2011b}
\CSLLeftMargin{66. }%
\CSLRightInline{Cascalheira, J., Bicho, N., Marreiros, J., \& Pereira,
T. (2011). \emph{Dating the Early Upper Paleolithic of Southwestern
Iberia: The case of Vale Boi}. ESHE 2011, Leipzig, DE.}

\bibitem[\citeproctext]{ref-bicho_2011c}
\CSLLeftMargin{67. }%
\CSLRightInline{Bicho, N., Cascalheira, J., Marreiros, J., \& Pereira,
T. (2011). \emph{Early Upper Paleolithic ecodynamics in southern
Iberia}. ESHE 2011, Leipzig, DE.}

\bibitem[\citeproctext]{ref-dias_2011a}
\CSLLeftMargin{68. }%
\CSLRightInline{Dias, R., Cascalheira, J., Gonçalves, C., Detry, C., \&
Bicho, N. (2011). \emph{Preliminary analysis of the spatial
relationships between faunal and lithic remains on the Mesolithic
shellmidden of Cabeço da Amoreira (Muge, Portugal)}. IV Jornadas do
Quaternário -- APEQ, Coimbra, PT.}

\bibitem[\citeproctext]{ref-pereira_2011a}
\CSLLeftMargin{69. }%
\CSLRightInline{Pereira, T., Bicho, N., Cascalheira, J., Marreiros, J.,
Gibaja, J., \& Haws, J. (2011). \emph{The impact of raw material
availability in the SW Iberian Paleolithic}. ESHE 2011, Leipzig, DE.}

\bibitem[\citeproctext]{ref-pereira_2011b}
\CSLLeftMargin{70. }%
\CSLRightInline{Pereira, T., Bicho, N., Cascalheira, J., Marreiros, J.,
Gibaja, J., \& Haws, J. (2011). \emph{Differences in the management of
raw material in the Upper and Middle Paleolithic in SW Iberia}. UKAS
2011, Reading, UK.}

\bibitem[\citeproctext]{ref-cascalheira_2011d}
\CSLLeftMargin{71. }%
\CSLRightInline{Cascalheira, J., Gonçalves, C., \& Bicho, N. (2011).
\emph{Shellmiddens as a landmark: Visibility analysis in the Mesolithic
of Muge valley (Central Portugal)}. UKAS 2011, Reading, UK.}

\bibitem[\citeproctext]{ref-bicho_2011f}
\CSLLeftMargin{72. }%
\CSLRightInline{Bicho, N., Manne, T., Marreiros, J., Cascalheira, J.,
Pereira, T., Tátá, F., Évora, M., \& Gonçalves, C. (2011). \emph{A tale
of two seas: Upper Paleolithic ecology in Vale Boi, Southwestern Algarve
(Portugal) and the arrival of the First Modern Humans to Southwestern
Iberia}. INQUA 2011, Bern, CH.}

\bibitem[\citeproctext]{ref-bicho_2011g}
\CSLLeftMargin{73. }%
\CSLRightInline{Bicho, N., Pereira, T., Cascalheira, J., Marreiros, J.,
Gonçalves, C., \& Dias, R. (2011). \emph{The chronology of the
Mesolithic occupation of the Muge valley, Central Portugal}. INQUA 2011,
Bern, CH.}

\bibitem[\citeproctext]{ref-marreiros_2011a}
\CSLLeftMargin{74. }%
\CSLRightInline{Marreiros, J., Cascalheira, J., \& Bicho, N. (2011).
\emph{Flake production in the Early Upper Paleolithic assemblages of
Vale Boi (Southern Portugal)}. Flakes not blades - Discussing the role
of flake production at the onset of the Upper Palaeolithic, Mettman,
DE.}

\bibitem[\citeproctext]{ref-jesus_2010}
\CSLLeftMargin{75. }%
\CSLRightInline{Jesus, L., Marreiros, J., Cascalheira, J., Gibaja, J.,
Pereira, T., \& Bicho, N. (2010). \emph{Occupation, functionality and
culture. Preliminary results from microliths technology and use-wear
analysis of Cabeço da Amoreira (Muge, Portugal)}. MESO 2010, Santander,
ES.}

\bibitem[\citeproctext]{ref-cascalheira_2010b}
\CSLLeftMargin{76. }%
\CSLRightInline{Cascalheira, J., Marreiros, J., \& Évora, M. (2010).
\emph{From Gravettian to Solutrean in Southwestern Iberia: A
technological perspective}. Paleoanthropological Society Annual Meeting,
Saint Louis, US.}

\bibitem[\citeproctext]{ref-bicho_2010c}
\CSLLeftMargin{77. }%
\CSLRightInline{Bicho, N., Manne, T., Marreiros, J., Cascalheira, J.,
Tátá, F., Gibaja, J., Évora, M., \& Gonçalves, C. (2010). \emph{On the
edge: The Upper Paleolithic from Vale Boi, Algarve, Portugal and the
arrival of the first modern humans to Southwestern Iberia}.
Paleoanthropological Society Annual Meeting, Saint Louis, US.}

\end{CSLReferences}

\section{\texorpdfstring{\ul{Fieldwork}}{Fieldwork}}\label{fieldwork}

\subsection{\texorpdfstring{\ul{As
coordinator}}{As coordinator}}\label{as-coordinator}

\begin{cvhonors}
    \cvhonor{}{Excavations at the Mesolithic site of Cabeço dos Ossos, Portugal}{}{2021}
    \cvhonor{}{Excavations at the Paleolithic site of Gruta do Escoural, Portugal}{}{2020-}
    \cvhonor{}{Excavations at the Paleolithic site of Gruta da Companheira, Portugal}{}{2017-}
    \cvhonor{}{Excavations at the Paleolithic site of Vale Boi, Portugal}{}{2009-2019}
    \cvhonor{}{Excavations at the Mesolithic site of Cabeço da Amoreira, Portugal}{}{2009-}
    \cvhonor{}{Excavations at the Mesolithic site of Cabeço da Arruda, Portugal}{}{2013}
    \cvhonor{}{Excavations at the Paleolithic sites of Gruta Nova da Columbeira and Lapa do Suão, Portugal}{}{2009}
\end{cvhonors}

\subsection{\texorpdfstring{\ul{As invited
specialist}}{As invited specialist}}\label{as-invited-specialist}

\begin{cvhonors}
    \cvhonor{}{DISPERSALS - Resilience, and innovation in Late Pleistocene SE Africa. Lithics specialist. PI: Nuno Bicho}{}{2023-}
    \cvhonor{}{Survey and excavations in Mozambique under the project Stone Age Vilankulos: Modern Human Origins Research south of the Rio Save, Mozambique. Lithics specialist and field manager. PI: Jonathan Haws}{}{2016-19}
    \cvhonor{}{Excavations at Dehesilla Cave, Spain. Total station specialist. PI: Daniel Garcia Rivero}{}{2016-19}
    \cvhonor{}{Survey and excavations in Mozambique under the project Middle Stone Age archaeology and the origins of modern humans in southern Mozambique. Lithics specialist and field manager. PI: Nuno Bicho}{}{2013-15}
    \cvhonor{}{Excavations at Picareiro Cave. Total station specialist. PI: Jonathan Haws}{}{2014-15}
\end{cvhonors}

\subsection{\texorpdfstring{\ul{Other
fieldwork}}{Other fieldwork}}\label{other-fieldwork}

\begin{cvhonors}
    \cvhonor{}{Excavations at the Mesolithic site of Cabeço da Amoreira, Portugal. Assistant field manager. PI: Nuno Bicho.}{}{2008}
    \cvhonor{}{Testing at the site of Zavial Rockshelter, Portugal. Excavator. PI: Nuno Bicho.}{}{2006}
    \cvhonor{}{Excavations at the Paleolithic site of Vale Boi, Portugal. Assistant field manager. PI: Nuno Bicho}{}{2005-06}
    \cvhonor{}{Excavations at the Mesolithic site of Barranco das Quebradas, Portugal. Excavator. PI: Maria J. Valente}{}{2004}
\end{cvhonors}

\section{\texorpdfstring{\ul{Teaching}}{Teaching}}\label{teaching}

\begin{cventries}
    \cventry{Courses taught:}{Invited Professor}{University of Coimbra, Portugal}{2018-2019}{\begin{cvitems}
\item Field Archaeology (MA in Human Evolution)
\end{cvitems}}
    \cventry{Courses taught:}{Invited Professor}{Universidade do Algarve, Portugal}{2014-Present}{\begin{cvitems}
\item Introduction to Archaeology (Undergrad. in Archaeology)
\item Archaeological Theory (MA in Archaeology)
\item Lithic analysis (MA in Archaeology)
\item Introduction to lithic analysis (Undergrad. in Archaeology)
\item Introduction to Data Analysis in Archaeology (Undergrad. in Archaeology)
\item Prehistoric adaptations to climate change (Ph.D. in Archaeology)
\item Digging data: quantitative and reproducible analysis in Archaeology (Ph.D. in Archaeology)
\end{cvitems}}
\end{cventries}

\section{\texorpdfstring{\ul{Administrative and management
experience}}{Administrative and management experience}}\label{administrative-and-management-experience}

\subsection{\texorpdfstring{\ul{Scientific and University
management}}{Scientific and University management}}\label{scientific-and-university-management}

\begin{cventries}
    \cventry{Interdisciplinary Center for Archaeology and the Evolution of Human Behaviour}{Director}{ICArEHB - Universidade do Algarve, Portugal}{2023-}{}\vspace{-4.0mm}
\end{cventries}\begin{cventries}
    \cventry{Quality Assurance Committee}{Invited member}{Universidade do Algarve, Portugal}{2023-}{}\vspace{-4.0mm}
\end{cventries}\begin{cventries}
    \cventry{Faculdade de Ciências Humanas e Sociais Scientific Council}{Member}{Universidade do Algarve, Portugal}{2023-}{}\vspace{-4.0mm}
\end{cventries}\begin{cventries}
    \cventry{Academic Senate}{Member}{Universidade do Algarve, Portugal}{2023-}{}\vspace{-4.0mm}
\end{cventries}\begin{cventries}
    \cventry{Research Council of the Unidade de Apoio à Investigação Científica e Formação Pós-Graduada}{Member}{Universidade do Algarve, Portugal}{2023-}{}\vspace{-4.0mm}
\end{cventries}\begin{cventries}
    \cventry{Research Line on Dynamics of Paleolithic people in Europe}{Coordinator}{ICArEHB - Universidade do Algarve, Portugal}{2022}{}\vspace{-4.0mm}
\end{cventries}\begin{cventries}
    \cventry{Research Line on African Archaeology and Human Evolution}{Coordinator}{ICArEHB - Universidade do Algarve, Portugal}{2019-2021}{}\vspace{-4.0mm}
\end{cventries}\begin{cventries}
    \cventry{UAlg's General Council}{Elected member}{Universidade do Algarve, Portugal}{2021-2024}{}\vspace{-4.0mm}
\end{cventries}

\subsection{\texorpdfstring{\ul{Search
committees}}{Search committees}}\label{search-committees}

\begin{cvhonors}
    \cvhonor{}{Call for 1 PhD Fellowship (BI) in Archaeology, expertise in climate modeling, within FINISTERRA: Population trajectories and cultural dynamics of Late Neanderthals in Far Western Eurasia (101045506-FINISTERRA-ERC-2021-COG). Notice nº 046/2023}{}{2023}
    \cvhonor{}{Call for 1 PhD Fellowship (BI) in Archaeology, expertise in geoarchaeology, within FINISTERRA: Population trajectories and cultural dynamics of Late Neanderthals in Far Western Eurasia (101045506-FINISTERRA-ERC-2021-COG). Notice nº 037/2023}{}{2023}
    \cvhonor{}{Call for 1 MA Fellowship (BI) in Archaeology at ICArEHB. Notice nº 001/2023}{}{2023}
    \cvhonor{}{Call for 3 PhD Fellowships (BI) in Archaeology, expertise in isotopes, phytoliths, computational modeling, within DISPERSALS: Dispersals, resilience, and innovation in Late Pleistocene SE Africa, ERC-2021-ADG. Notice nº 039/2023}{}{2023}
    \cvhonor{}{Call for 1 Principal Researcher in Archaeology within DISPERSALS: Dispersals, resilience, and innovation in Late Pleistocene SE Africa, ERC-2021-ADG. Notice nº 15831/2023}{}{2023}
    \cvhonor{}{Call for 1 Post-Doc Fellowship (BI) in Archaeology, within DIASPORA: Early Human migrations and the Nile Valley: the Kerma region during the MSA. Notice nº 071/2022}{}{2023}
    \cvhonor{}{Call for 4 Auxiliar Researchers in Archaeology, expertises in isotopes, phytoliths, computational modeling, biomarkers, within DISPERSALS: Dispersals, resilience, and innovation in Late Pleistocene SE Africa, ERC-2021-ADG. Notice n.º 17876/2022}{}{2022}
    \cvhonor{}{Call for 1 post-doctoral fellowship in Archaeology, expertise in fauna taphonomy, within FINISTERRA: Population trajectories and cultural dynamics of Late Neanderthals in Far Western Eurasia (101045506-FINISTERRA-ERC-2021-COG). Notice nº 070/2022}{}{2022}
    \cvhonor{}{Call for 1 post-doctoral Assistant Researcher contract in Archaeology at ICArEHB, through CEEC institutional program. Notice nº 202205/0517}{}{2022}
    \cvhonor{}{Call for 1 PhD Fellowship (BI) in Archaeology, expertise in lithic analysis, within FINISTERRA: Population trajectories and cultural dynamics of Late Neanderthals in Far Western Eurasia (101045506-FINISTERRA-ERC-2021-COG). Notice nº 034/2022}{}{2022}
    \cvhonor{}{Call for 1 PhD Fellowship (BI) in Archaeology, expertise in microfauna analysis, within FINISTERRA: Population trajectories and cultural dynamics of Late Neanderthals in Far Western Eurasia (101045506-FINISTERRA-ERC-2021-COG). Notice nº 035/2022}{}{2022}
    \cvhonor{}{Call for 1 post-doctoral fellowship in Archaeology, expertise in isotope analysis, within FINISTERRA: Population trajectories and cultural dynamics of Late Neanderthals in Far Western Eurasia (101045506-FINISTERRA-ERC-2021-COG). Notice nº 039/2022}{}{2022}
    \cvhonor{}{Call for 1 post-doctoral Assistant Researcher contract in Archaeology at ICArEHB, through CEEC institutional program}{}{2022}
    \cvhonor{}{Call for 1 MA fellowship in Archaeology within the project ALG-45-2020-41– SciTour - Turismo Científico: uma nova abordagem para promover o turismo no Algarve}{}{2022}
    \cvhonor{}{Call for 1 Research Fellowship (BI) in Archaeology within the project DIASPORA – As primeiras migrações humanas no vale do Nilo: a região de Kerma durante o MSA (PTDC/HAR-ARQ/0131/2020). Notice nº 024/2022}{}{2022}
    \cvhonor{}{Call for 1 Research Fellowship (BI) in Archaeology at ICArEHB (UID/ARQ/04211/2019). Notice nº 057/2021}{}{2021}
    \cvhonor{}{Call for 1 Research Fellowship (BI) in Archaeology within The Muge Shellmiddens Project: A new Portal to the last hunter-gatherers the Tagus Valley, Portugal (ALG-01-0145-FEDER-29680). Notice nº 033/2021.}{}{2021}
    \cvhonor{}{Call for 4 Ph.D. fellowships (BD) in Prehistoric Archaeology at ICArEHB (UID/ARQ/04211/2019)}{}{2021}
    \cvhonor{}{Call for 1 post-doctoral contract in Archaeology at ICArEHB (UID/ARQ/04211/2019)}{}{2020}
    \cvhonor{}{Call for 1 Research Fellowship (BI) in 3D Animation within The Muge Shellmiddens Project: A new Portal to the last hunter-gatherers the Tagus Valley, Portugal (ALG-01-0145-FEDER-29680). Notice nº 079/2019}{}{2020}
    \cvhonor{}{Call for 1 Research Fellowship (BI) in Computer Science and engineering within The Muge Shellmiddens Project: A new Portal to the last hunter-gatherers the Tagus Valley, Portugal (ALG-01-0145-FEDER-29680). Notice nº 001/2020}{}{2020}
    \cvhonor{}{Call for 1 Research Fellowship (BI) in 3D Animation within The Muge Shellmiddens Project: A new Portal to the last hunter-gatherers the Tagus Valley, Portugal (ALG-01-0145-FEDER-29680). Notice nº 002/2020}{}{2020}
    \cvhonor{}{Call for 1 Research Fellowship (BI) in Design within The Muge Shellmiddens Project: A new Portal to the last hunter-gatherers the Tagus Valley, Portugal (ALG-01-0145-FEDER-29680). Notice nº 018/2020}{}{2020}
    \cvhonor{}{Call for 1 Research Fellowship (BI) in Archaeology within The origins and evolution of human cognition and the impact of Southwestern European coastal ecology (ALG-01-0145-FEDER-27833). Notice nº 060/2020}{}{2020}
    \cvhonor{}{Call for 1 funding officer position at ICArEHB (UID/ARQ/04211/2019)}{}{2020}
    \cvhonor{}{Call for 1 administrative position at ICArEHB (UID/ARQ/04211/2019). Notice nº 046/2019}{}{2019}
    \cvhonor{}{Call for 1 post-doctoral contract within the project FLAME – Fire among Anatomically Modern Humans and Neanderthals as revealed from Archaeological Microstratigraphic evidence}{}{2019}
    \cvhonor{}{Call for 1 post-doctoral contract within the project The origins and evolution of human cognition and the impact of Southwestern European coastal ecology (ALG-01-0145-FEDER-27833)}{}{2019}
\end{cvhonors}

\section{\texorpdfstring{\ul{Extension and knowledge
dissemination}}{Extension and knowledge dissemination}}\label{extension-and-knowledge-dissemination}

\subsection{\texorpdfstring{\ul{Student and Postdoctoral
supervision}}{Student and Postdoctoral supervision}}\label{student-and-postdoctoral-supervision}

\begin{cvhonors}
    \cvhonor{}{Nolan Ferar (Ph.D. dissertation, ICArEHB, Universidade do Algarve, Portugal) - Funded by FCT}{}{2022-}
    \cvhonor{}{Mpumi Maringa (Ph.D. dissertation, ICArEHB, Universidade do Algarve, Portugal}{}{2022-}
    \cvhonor{}{Jovan Galfi (Ph.D. dissertation, ICArEHB, Universidade do Algarve, Portugal}{}{2020-}
    \cvhonor{}{Joana Belmiro (Ph.D. dissertation, Universidade do Algarve, Portugal) - Funded by FCT}{}{2020-}
    \cvhonor{}{Daniela Maio (Ph.D. dissertation, Universidade do Algarve, Portugal)}{}{2019-}
    \cvhonor{}{Emily Hallinan (Marie Curie Fellow, ICArEHB, Universidade do Algarve}{}{2020-2022}
    \cvhonor{}{Pedro Horta (Ph.D. dissertation, Universidade do Algarve, Portugal) - Funded by FCT}{}{2017-2022}
    \cvhonor{}{Joana Belmiro (M.A. dissertation, Universidade do Algarve, Portugal)}{}{2017-2019}
    \cvhonor{}{Joana Belmiro (Undergrad. dissertation, Universidade do Algarve, Portugal)}{}{2017}
    \cvhonor{}{Pedro Horta (M.A. dissertation, Universidade do Algarve, Portugal)}{}{2014-15}
\end{cvhonors}

\subsection{\texorpdfstring{\ul{Invited
seminars}}{Invited seminars}}\label{invited-seminars}

\begin{cvhonors}
    \cvhonor{}{``The Upper Paleolithic occupations at Vale Boi (Vila do Bispo, Algarve)''. Associação Arqueológica do Algarve, São Brás de Alportel/Lagoa}{}{2024}
    \cvhonor{}{``40 mil anos depois: o que sabemos sobre o desaparecimento dos Neandertais''. Academia das Ciências de Lisboa, Lisbon}{}{2023}
    \cvhonor{}{``O que sabemos sobre o desaparecimento dos Neandertais?''. FI.CA, Lisbon}{}{2022}
    \cvhonor{}{``Human adaptations to the Last Glacial Maximum in the Iberian Peninsula''. Hebrew University MA seminar series ``The social dimensions of human adaptation to extreme environments'', Online}{}{2021}
    \cvhonor{}{``Proto-Solutrean and Solutrean adaptations in westernmost Iberia: the Vale Boi and Lapa do Picareiro case-studies''. Erlangen, DE (with Nuno Bicho, Joana Belmiro and Jonathan Haws)}{}{2019}
    \cvhonor{}{``A última decada de trabalhos arqueológicos no concheiro do Cabeço da Amoreira, Muge''. Muge, Salvaterra de Magos, PT (with Célia Gonçalves, Nuno Bicho and Lino André)}{}{2019}
    \cvhonor{}{``Evolução humana e a tecnologia dos instrumentos em pedra''. Universidade de Coimbra, Coimbra, Portugal, Universidade de Coimbra, Coimbra, PT.}{}{2017}
    \cvhonor{}{``Neanderthal to Modern Humans: the transition in Portugal''. Algarve Archaeological Association, São Brás de Alportel, PT}{}{2017}
    \cvhonor{}{``Prehistoric Hunter-Gatherer Adaptations in Southwestern Iberia and Mozambique''. Max Planck Institute for the Science of Human History, Jena, DE}{}{2016}
    \cvhonor{}{``The end of the Middle Paleolithic and the arrival of anatomically modern humans to Southwestern Iberia''. Max Planck Institute, Leipzig, DE (with Nuno Bicho)}{}{2016}
    \cvhonor{}{``Recent developments in the analysis of lithic technology''. University of Seville, ES}{}{2016}
    \cvhonor{}{``Laboratorial techniques for the analysis of lithic industries''. University Eduardo Mondlane, Maputo, MZ}{}{2015}
    \cvhonor{}{``Abrupt climate change and human adaptations''. University Eduardo Mondlane, Maputo, MZ}{}{2014}
    \cvhonor{}{``Stone Age archaeology in the Niassa region, Mozambique. Preliminary results from the 2014 campaign''. Geographical Society of Lisbon, Lisbon, PT (with N. Bicho, J. Haws, O. Madime, C. Gonçalves, V. Aldeias, and M. Benedetti)}{}{2014}
    \cvhonor{}{``Human responses to environmental change: the case of the Portuguese Upper Paleolithic''. National Museum of Archaeology, Lisbon, PT}{}{2014}
    \cvhonor{}{``The mediterranean influence on the social networks of the Solutrean in Iberia''. National Museum of Archaeology, Lisbon, PT}{}{2014}
\end{cvhonors}

\subsection{\texorpdfstring{\ul{Conferences/Symposia
Organized}}{Conferences/Symposia Organized}}\label{conferencessymposia-organized}

\begin{cvhonors}
    \cvhonor{}{``Symposium Expedient Technological Behavior: Global Perspectives and Future Directions''. With Nolan Ferar. SAA 2024, New Orleans, USA}{}{2024}
    \cvhonor{}{``European Society for Human Evolution Annual Meeting''. Universidade do Algarve, Faro, PT}{}{2018}
    \cvhonor{}{``International Congress The Solutrean''. With Isabell Schmidt, Nuno Bicho and Gerd-Christian Weniger. Universidade do Algarve, Faro, PT}{}{2017}
    \cvhonor{}{``Symposium Global perspectives on the impact of abrupt climate change in hunter-gatherer-technologies''. With Nuno Bicho. SAA 2016, Orlando, USA}{}{2016}
    \cvhonor{}{Conference cycle ``Arqueologia às 6''. Organized by ICArEHB. Faro, PT}{}{2015-2017}
    \cvhonor{}{``Symposium Solutrean points: past and present''. With Juan Gibaja Bao. At the International Conference El Solutrense. Almería, ES}{}{2012}
    \cvhonor{}{``IV International Conference of Young Researchers in Archaeology (JIA 2011)''. Universidade do Algarve, Faro, PT}{}{2011}
    \cvhonor{}{``Session Demographic processes and cultural change: archaeological perspectives''. III International Conference of Young Researchers in Archaeology (JIA 2010). Barcelona, ES}{}{2010}
\end{cvhonors}

\subsection{\texorpdfstring{\ul{Workshops and Exhibitions
Organized}}{Workshops and Exhibitions Organized}}\label{workshops-and-exhibitions-organized}

\begin{cvhonors}
    \cvhonor{}{Workshop ``Data Analysis and R Statistics''. Universidade do Algarve, Faro, PT}{}{2016}
    \cvhonor{}{Exhibition ``The Prehistoric Origins of the Algarve''. Organized by ICArEHB and Centro de Ciência Viva do Algarve. Faro, Tavira, Lagos and Vila do Bispo. PT}{}{2016}
    \cvhonor{}{Exhibition ``At the End of the World 30 thousand years ago: the prehistoric hunter-gatherers of Vale Boi''. Organized by ICArEHB. Vila do Bispo, PT}{}{2014}
    \cvhonor{}{Workshop ``Total Station in Archaeology''. Universidade do Algarve, PT. Promoted by NAP}{}{2011}
    \cvhonor{}{Exhibition ``On the margins of the past: the Muge shell middens''. Salvaterra de Magos, PT. Organized by NAP}{}{2011}
    \cvhonor{}{Conferences ``Arqueologia ao Sul''. Universidade do Algarve, Faro, PT. Organized by NAP}{}{2010-14}
    \cvhonor{}{Workshop ``Analysis of Lithic Industries'' (Universidade do Algarve, Faro, PT. Promoted by NAP}{}{2010}
    \cvhonor{}{Workshop ``NESPOS On Tour''. Universidade do Algarve, Faro PT. Promoted by NAP}{}{2010}
\end{cvhonors}

\subsection{\texorpdfstring{\ul{Tutoring}}{Tutoring}}\label{tutoring}

\begin{cvhonors}
    \cvhonor{}{Instructor of Statistics in Archaeology using R in one the Training in Frontier Archaeology ICArEHB actions}{}{2022}
    \cvhonor{}{Creation of the archaeological survey and excavation Massive Open Online Courses (MOOC) modules of the *Online Learning on African Archeology and Heritage (ONLAAH)* project}{}{2019-}
\end{cvhonors}

\section{\texorpdfstring{\ul{Additional
Training}}{Additional Training}}\label{additional-training}

\begin{itemize}
\tightlist
\item
  \ul{Workshop \emph{Introduction to Open Science.} Universidade do
  Algarve, Faro, PT.}
\item
  \ul{Workshop \emph{Machine Learning in Archaeology}. Universidade do
  Algarve, Faro, PT:}
\item
  \ul{Workshop \emph{Science and public policy: how to build bridges?}.
  Universidade do Algarve, Faro, PT.}
\item
  \ul{Workshop \emph{Data Analysis and R Statistics}. Universidade do
  Algarve, Faro, PT.}
\item
  \ul{Workshop \emph{EVAN Toolbox}. Max Planck Institute for
  Evolutionary Abthropology, Leipzig, DE.}
\item
  \ul{Workshop \emph{NESPOS on Tour}. Universidade do Algarve, Faro,
  PT.}
\item
  \ul{Workshop \emph{Understanding tool use, multidisciplinary
  perspectives on the cognition and ecology of tool using behaviors}.
  Max Planck Institute for Evolutionary Anthropology, Leipzig, DE.}
\item
  \ul{Workshop \emph{Tools in organic materials: methods of analysis}.
  CSIC, Barcelona, ES.}
\item
  \ul{Workshop \emph{Advanced SPSS applications}. Universidade do
  Algarve, Faro, PT.}
\item
  \ul{Workshop \emph{Functional studies in non-siliceous rocks}. Lisbon,
  PT.}
\end{itemize}

\section{\texorpdfstring{\ul{Reviewer for academic journals and funding
agencies}}{Reviewer for academic journals and funding agencies}}\label{reviewer-for-academic-journals-and-funding-agencies}

\begin{itemize}
\tightlist
\item
  \ul{Fund for Scientific Research -- FNRS Belgium (1)}
\item
  \ul{Nature Scientific Reports (1)}
\item
  \ul{PLoS ONE (4)}
\item
  \ul{Journal of Human Evolution (2)}
\item
  \ul{Advances in Archaeological Practice (2)}
\item
  \ul{Journal of Archaeological Science (1)}
\item
  \ul{Journal of Archaeological Science: Reports (2)}
\item
  \ul{Quaternary International (1)}
\item
  \ul{Journal of Paleolithic Archaeology (3)}
\item
  \ul{African Archaeological Review (1)}
\item
  \ul{Antiquity (1)}
\item
  \ul{National Science Centre Poland (1)}
\item
  \ul{Paleoanthropology Journal (1)}
\item
  \ul{Journal of Open Archaeology Data (1)}
\item
  \ul{Journal of Environmental Archaeology (1)}
\end{itemize}

\section{\texorpdfstring{\ul{Professional
memberships}}{Professional memberships}}\label{professional-memberships}

\begin{itemize}
\tightlist
\item
  \ul{European Society for the Study of Human Evolution}
\item
  \ul{Society for American Archaeology}
\item
  \ul{Paleoanthropology Society}
\item
  \ul{Archaeological Institute of America}
\item
  \ul{Seminário Permanente dos Jovens Cientistas do Instituto de Altos
  Estudos da Academia das Ciências de Lisboa, Membro.}
\end{itemize}

\section{\texorpdfstring{\ul{Languages}}{Languages}}\label{languages}

\begin{itemize}
\tightlist
\item
  \ul{Portuguese (native)}
\item
  \ul{English (proficient)}
\item
  \ul{Spanish (conversational)}
\end{itemize}



\end{document}
